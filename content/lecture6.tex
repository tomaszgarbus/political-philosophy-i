%%%%%%%%%%%%%%%%%%%%%%%%%%%%%%%%%%%%%%%%%%%%%%%%%%%%%%%%%%%%%%%%%%%%%%%%%%%%%%

\section{Lecture 6: Human Rights}

\subsection{Natural law}

\textbf{Political philosophy}

\begin{itemize}
    \item Premodern political philosophy: politics is natural (part of the
    natural/religious world).
    \item Modern political philosophy: politics is conventional.
    \item But there are "natural rights" whose source is not politics but
    human nature (or God).
    \item That is, there are basic norms that are universal and "come before"
    politics.
    \item They can be discovered by reason (rationality).
\end{itemize}

\textit{Every man ought to endeavour peace, as far as he has hope of
obtaining it; and when he cannot obtain it, that he may seek and use all
helps and advantages of war.} (Hobbes)

\textit{Reason, which is that law, teaches all mankind, who will but consult
it, that being all equal and independent, no one ought to harm another...}
(Locke)

\subsection{Declaration of the Rights of Man and of the Citizen (1789)}

\textbf{Article I} Men are born and remain free and equal and rights. Social
distinctions may be founded only upon the general good.

\textbf{Article II} The goal of any political association is the conservation
of the natural and impresciptible rights of man. These rights are liberty,
property, safety and resistance against oppression.

...

Thomas Paine, The rights of man (1791-1792):
\textit{Natural rights are those which appertain to man in right of his
existence... Civil rights are those which appertain to man in right of being
a member of society. Every civil right has for its foundation, some natural
right pre-existing in the individual, but to the enjoyment of which his
individual power is not, in all cases, sufficiently competent.}
\begin{itemize}
    \item Rights of the citizen $\rightarrow$ political rights
    $\rightarrow$ civil rights
    \item Rights of man $\rightarrow$ natural rights (moral rights)
    $\rightarrow$ human rights
\end{itemize}

\subsection{The Universal Declaration of Human Rights (1948)}

\begin{itemize}
    \item All human beings are born free and equal in dignity and rights.
    \item Everyone has the right to life, liberty and security of person.
    \item No one shall be held in slavery or servitude...
    \item No one shall be subjected to torture or to cruel, inhuman or
    degrading treatment or punishment.
    \item Everyone has the right to recognition everywhere as a person before
    the law.
\end{itemize}

\subsection{Theories of human rights}

\textbf{Natural law tradition}: Human rights are the natural rights (moral
rights based on the laws of nature) of the Western Enlightenment political
philosophy tradition.

\textbf{Quietism}: \textit{We agree about the rights but on condition that
no one asks us why.} (Jacques Maritain)

\textbf{Interest-based accounts}: Human rights are formulations or
expressions of fundamental human interests.

\textit{[I reject] the idea of abstract right as a thing independent of
utility. I regard utility as the ultimate appeal on all ethical questions;
but it must be utility in the largest sense, grounded on the permanent
interests of man as a progressive being.} (Mill)

\textbf{Practical accounts}: Human rights as the currency of "public
reasons", that is principles and norms that all decent political communities
accept. (To account for them, you must start from looking at the practice
of human rights.)

\textbf{Agency-based accounts}: Human rights are protections of human agency
(being an autonomous actor or decision maker).

\textbf{Institutional accounts}: Human rights are requirements of justice
(you have to first work out a theory of justice to account for them).

\textbf{Agnosticism}: There may be no single justification of human rights.
They can be justified on different theoretical bases. (E.g. natural law
theory, utilitarianism, agency-based accounts, justice, ...). What matters is
the general acceptance and political implications.

\subsection{The metaethical commitments of human rights}

What are the in-built assumptions of human rights?

\textbf{Moral realism}: There are moral facts that make moral judgements true
or false (objectivity). (When one says people have moral rights, one
expresses a fact.)

\textbf{Cognitivism}: Moral judgments express beliefs. Moral disagreements are
genuine disagreements. (Disagreements about human rights are genuine
disagreements that can be resolved).

\textbf{Anti-relativism}: Moral reasons apply to everyone. (Human rights are
universally valid, that is, possessed by everyone.)

\subsection{Tidying up}

A frequent (and not entirely invalid) criticism of the development of human
rights is that they are too expansive:
\begin{itemize}
    \item Everyone has the right to life, liberty and security of person.
    \item Everyone has the right to freedom of opinion and expression.
    \item Everyone has the right to rest and leisure, including reasonable
    limitation of working hours and periodic holidays with pay.
\end{itemize}

\begin{itemize}
    \item Core human rights: articles 1-5
    \item Civil and political rights: ca. articles 6-21
    \item Social rights: articles 22-29
\end{itemize}

\subsection{Are human rights universal?}

American Anthropological Association, Statement on Human Rights (1947)
\textit{How can the proposed Declaration be applicable to all human beings,
and not be a statements of rights conceived only in terms of the values
prevalent in the countries of Western Europe and America? ... Standards and
values are relative to the culture from which they derive so that any
attempt to formulate postulates that grow out of the belief or moral codes
of one culture must to that extent detract from the applicability of any
Declaration of Human Rights to mankind as a whole.}

\subsection{Are there multiple human rights "models"?}

\subsection{What are the features of human rights?}

\textbf{Are human rights conditional?} Unconditional

\textbf{Can human rights be alienated?} Inalienable

\textbf{Do they apply to everyone?} Universal

\textbf{Are human rights granted by the state?} Prepolitical

\textbf{Who can violate human rights?} Institutional

\subsection{What are the features of human rights?}

\textbf{The asymmetrical nature of human rights}
\begin{itemize}
    \item Having a right entails that there is a corresponding obligation:
    an obligation holder that has to respect or fulfill the right.
    \item Human rights are not granted by political authority.
    \item But the obligation holder is political authority: violations of
    human rights are acts committed in an official capacity.
\end{itemize}

\subsection{Human rights and history}

\textbf{Did Athens violate human rights by allowing slavery?}
\begin{itemize}
    \item Human rights are not granted by the state or political authority;
    they are independent of social or political recognition.
    \item Human rights do not vary with social, cultural historical, or
    economic circumstances: they apply to all humans in all political
    societies.
\end{itemize}

Socrates says:
\textit{We should listen to some opinions but not others. We should value the
good opinions, but not the bad ones. We should listen only to the expert on
justice and injustice, to the one man, and to truth itself.}

\begin{itemize}
    \item We are autonomous moral subjects who should care about
    justifications.
    \item If we are to accept political authority, we are owed a
    justification.
\end{itemize}

\subsection{Human rights and political authority}

\begin{enumerate}
    \item Political authority demands that you take its command as
    exclusionary reasons.
    \item This demand is justified by the claim that the holder of political
    authority is entitled to rule. (Its source of authority is not mere
    power.)
    \item The claim that the holder of political authority is entitled to
    rule is the claim that its rule is morally justified.
    \item If its entitlement to rule is morally justified, than its
    justification can be \textbf{addressed to you} (over whom the
    political authority rules).
    \item If that justification can be addressed to you, then you must be
    an autonomous moral subject (Slaves are not given justifications).
    \item You can only be an autonomous moral subject if you have basic
    rights.
\end{enumerate}

\textbf{Political authority}

Political power that claims legitimacy (the moral entitlement to issue
authoritative commands, that is, to provide exclusionary reasons) presents
itself as being \textbf{morally justified} towards its subjects:

\begin{itemize}
    \item Political power necessarily appeals to some form of consent
    or endorsement or acceptance of its subjects.
    \item But this assumes that subjects are in a position to provide that
    consent or endorsement or acceptance.
    \item But this is incompatible with the idea that the life of the
    subjects is not secure, that they are in slavery or subjected to torture
    or inhuman treatment or lack a right of recognition before the law.
\end{itemize}

%%%%%%%%%%%%%%%%%%%%%%%%%%%%%%%%%%%%%%%%%%%%%%%%%%%%%%%%%%%%%%%%%%%%%%%%%%%%%%

%%%%%%%%%%%%%%%%%%%%%%%%%%%%%%%%%%%%%%%%%%%%%%%%%%%%%%%%%%%%%%%%%%%%%%%%%%%%%%%%
\section{Lecture 1: Plato - political obligation}

Book for the course: \textit{Political Philosophy} by Jean Hampton.

Handouts posted on Athena after each lecture.

On the forum, try to discuss with other posters instead of just posting your
own essay. Short and concise answers more appreciated.

Regular quizzes will be posted on Athena.

Key concepts and ideas listed on last page on each handout.

Exam:

\begin{itemize}
    \item 10 multiple choice questions
    \item Five short explanations and definitions
    \item Two 500-word essay questions
    \item The exam will be based only on material discussed in class
\end{itemize}

Socrates was sentenced to death for corrupting the youth. He was part of the
losing party during political turmoil, so it was kind of political vendetta
against Socrates.

Did Socrates have an \textit{obligation} to accept his sentence and to refuse
the chance to escape?

\subsection{Power and authority}

\textbf{Power}: X has power over Y if X can compel Y to perform some action
$p$.
\begin{itemize}
    \item X's power over Y gives Y \textit{a reason} to obey, but it does not
    create an \textit{obligation}
\end{itemize}

\textbf{Authority}: X has authority if X occupies a social role (e.g. in virtue
of their expertise) from which they can provide reasons for Y to act in certain
ways.

\begin{itemize}
    \item X's authority gives Y \textit{a reason} to obey, but it does not
    create an \textit{obligation}.
\end{itemize}

\textbf{Political authority}:
X has political authority over Y iff the fact that X requires Y to perform some
action $p$ gives Y a reason to perform $p$, regardless of what $p$ is, and
where this reason purports to override all (or perhaps nearly all) reasons Y
may have not to perform $p$.

\textbf{Preemptive (or exclusionary) reasons}:
Commands issued by political authority provide reasons that \textit{preempt or
override} other reasons. Political authority requires surrender of judgment.

\begin{itemize}
    \item It is the \textit{source of the command} that creates the reason,
    not its content or relation to other reasons
    \item X's political authority over Y gives Y an overriding reason = 
    \textit{obligation to obey}.
    \item Political authority can provide preemptive reasons because it has
    \textit{entitlement to rule}.
\end{itemize}

\subsection{The source of political authority}

\textbf{Natural subordination theories}

The nature of some creatures is such that they instinctively submit and take
direction from other beings whose natures fit them for dominance, rule,
and power.

\begin{itemize}
    \item The \textit{natural roles} argument for natural subordination
    \begin{itemize}
        \item Aristotle's natural subordination theory
        \begin{itemize}
            \item The natural domination of master over slave who has lower
            cognitive abilities
            \item The natural domination of men over women
        \end{itemize}
    \end{itemize}
    \item The consequentialist argument for natural subordination
    \begin{itemize}
        \item Enlightened colonialism (e.g. Mill)
    \end{itemize}
\end{itemize}

\subsection{Divine authority theories}

A ruler has legitimate political authority iff his authority comes in some way
from the authority possessed by God(s) whose rule over human beings is 
unquestionable.

\begin{itemize}
    \item \textbf{Ruler is (a) God}, a divine authority himself
    \begin{itemize}
        \item Egyptian pharaohs (Tutankhamum means \textit{living image
        of Amun} the Sun god)
        \item Dalai Lamas (re-incarnations of Avalokitesvara, \textit{
        the lord who looks down})
    \end{itemize}
    \item \textbf{Ruler is descended from God(s)} and has divine status due to
    this relationship
    \begin{itemize}
        \item Some Roman emperors (Augustus was \textit{Divi filius},
        \textit{Son of the Divine One})
        \item Japanese emperors (descendants of the goddess Amaterasu,
        the goddess of the Sun)
    \end{itemize}
\end{itemize}

\textbf{Divine right view}

Rulers are human but have been given the authority to rule by God (indirect
authorization).

\begin{itemize}
    \item Adam was given authority to rule the Earth by God and kings are
    his first-in-line descendants
    \item Robert Filmer (1588-1653), \textit{Patriarcha, or the Natural
    Power of Kings} (1680)
\end{itemize}

\textbf{Metaphysics and politics}

Natural subordination and divine authority theories are \textit{metaphysical}:
politics is part of the natural world.

\textbf{Politics as a moral problem}

\begin{itemize}
    \item If political authority is part of the natural order of the world,
    politics is not a moral problem (it is a question of metaphysics).
    \begin{itemize}
        \item There were medieval books guidelines for rulers which instructed
        how to be a good political leader, not from perspective of being good
        to your subject but to satisfy God.
    \end{itemize}
    \item If political authority is not part of the natural order, it must be
    explained how and why rulers can have it, and why subjects have an 
    obligation to obey it -- politics becomes a moral problem.
\end{itemize}

\textbf{The single most consequential idea of Western civilization:}
\textit{Politics is conventional}

\subsection{Conventional views of political obligation}

\textbf{Natural duty accounts}

There is a general non-voluntary obligation to maintain and promote just
institutions.

\textit{Do you think that a state can exist and survive in which the
decisions of law have no power, where they are ignored by citizens?}

\begin{itemize}
    \item How to explain your special obligations to your own state if
    political obligations are general?
    \item How are you bound to \textit{your} state?
\end{itemize}

\textbf{Associative accounts}

There are special, non-voluntary (role-) obligations towards one's political
institutions created by social roles or identities (e.g. obligation from
gratitude).

\textit{since you were brought into the world and nurtured and educated by
us, can you deny in the first place that you are our child and slave, as
your fathers were before you= And if this is true, you are not on equal terms
with us... Just as you may do no violence to your father or mother, much less
may you do violence to your country.}

\begin{itemize}
    \item Why does a social role or identity create political obligations
    in itself?
    \item How could the duties associated with a social role or identity that 
    is morally indefensible be morally binding?
\end{itemize}

\textbf{Transactional accounts}

There are special, non-voluntary obligations towards political institutions,
based on the requirement of reciprocity: political obligations are a matter
of fairness.

\textit{Not only have we ... given you and every other citizen a share in
every good that we have to offer, but we have even granted you and every
Athenian the right that if you do not like us when you have come of age, you
may go where you please and take your goods with you. None of our laws will
forbid it or interfere with you. Anyone who does not like us, the laws and
the state, and who wants to go to a colony or to any other city, may go where
he likes, and take his goods with him.}

\begin{itemize}
    \item Fairness is owed to fellow citizens; how do duties to fellow
    participants in a cooperative scheme establish obligations to the state?
\end{itemize}

\textbf{Social contract theories}

Political obligations arise from voluntary acts (consent) either explicitly
or implicitly.

\textit{Anyone who has seen the way in which we keep justice and administer
the state, and who remains here, has entered into a contract that he will
do as we command him. And if he disobeys us, he wrongs us... because he made
an agreement with us that he will obey our commands.}

\begin{itemize}
    \item Does "choosing to remain" constitute explicit or implicit consent?
    \item If consent is hypothetical, how can it create real-life obligations?
\end{itemize}

\subsection{Are all political authorities entitled to rule?}

\textbf{Political legitimacy}

Political legitimacy is \textit{justified} entitlement to rule.

Two different views:
\begin{itemize}
    \item \textbf{Political authority presupposes legitimacy.}
    Only legitimate political authority creates political obligations.
    That is, only \textit{legitimate governments have political authority}.
    Illegitimate governments have only power.
    \item \textbf{Political authority and legitimacy are distinct.}
    Even illegitimate governments have political authority. That is, they
    are entitled to rule, even if their entitlements is not
    all-things-considered morally justified.
\end{itemize}

\subsection{Are there legitimate states?}

\textbf{Philosophical anarchism}

There are no preemptive reasons and hence no political authority. The
authority of the state is not unique (it may have authority, but not
political authority).

\begin{itemize}
    \item Not to be confused with political anarchism, a view about social
    and political organization.
\end{itemize}
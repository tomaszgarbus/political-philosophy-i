\section{Key concepts 4}

\begin{itemize}
	\item \textbf{popular sovereignty}: (the general will) is concerned
	 with general matters (making laws)
	\item \textbf{self-sufficiency (state of nature)}: for Rousseau, people
	 in the state of nature are self-sufficient: there is no community
	 between them
	\item \textbf{natural versus political rights}: natural rights are
	 norms of reason that specify the general preconditions for human
	 existence and survival, they conform to human nature. Political rights
	 are particular social rules, laws and relations, arising from power
	 relations
	\item \textbf{pity} (empathy) the ability to refrain from harming
	 others
	\item \textbf{perfectibility of humans}: psychological adaptability
	\item \textbf{amour propre}: as humans emerge from the state of nature
	 (through more frequent human contact), their self-love develops into
	 amour propre: a form of love of self that is the function of one's
	 esteem by others, determined by pride, envy, jelaousy, greed
	\item \textbf{general will}: it is always right, it "wills" the
	 general good, the common will of citizens concerned not with pursuing
	 their own interests but with the well-being of society. Example is
	 building a dam.
	\item \textbf{will of all}: what the majority of people want, it may
	 be irrational and fail to recognize the general will
	\item \textbf{obeying the general will is "forced to be free"}: total
	 alienation of rights and powers to become the part of the Sovereign,
	 where the Sovereign represents the general will which is perfectly
	 aligned with everyone with every citizen concerned with the well-being
	 of the society
	\item \textbf{direct democracy}: danger of interference from popular
	 sovereignty
	\item \textbf{totalitarianism}: (Rousseau's totalitarianism) when
	 entering civil society, a second psychological transformation takes
	 place and people's will never conflicts with the general will.
	 People's "real" interests and freedom is in the following the general
	 will, even if they need to be coerced.
\end{itemize}

\section{Lecture 7: Justice -- Rawls}

\subsection{Classifcal liberalism}

\textbf{Mill's liberty (harm) principle}:
\textit{The only purpose for which power can be rightfully exercised over any
member of a civilized community, against his will, is to prevent harm to
others.}

\textbf{Rawl's principle of liberty}:
\textit{Each person has an equal right to the most extensive scheme of basic
liberties that is compatible with a similar scheme of liberties for others.}

\subsection{Utilitarianism}

\subsection{Welfare economics}

\subsubsection{Pareto optimality}

In the absence of interpersonal comparisons of utility, welfare economics have
only \textbf{limited normative principles} to evaluate social states.

$\forall_{i, x, y} x \succeq_i y \land \exists_{j \neq i} x \succeq_j
\rightarrow x \succ y$

\subsubsection{Pareto improvement}

A policy change that makes some individuals better off without making anybody
worse off.

\subsubsection{Pareto efficiency}

A social state in which no further Pareto improvements are possible.

\subsubsection{Pareto principle} The normative principle that Pareto
improvements are desirable to implement.

\subsection{Welfare economics}

Pareto optimality does not provide a complete ordering of social states.

\subsection{The original position}

\subsubsection{Submit to the veil of ignorance}
\begin{itemize}
	\item you do not know your morally irrelevant characteristics
	\item you are not concerned with others' interests
	\item you do not know your own conception of the good
	\item you do not know your position in society
	\item thus, you become a "moral person"
\end{itemize}

\textbf{Moral person:} rational, neither risk-averse nor risk-seeking person
with a sense of justice, able to act in accordance to her own conception of the
good, not influenced by envy, has a life plan, able to assess her goals from a
moral point of view.
\begin{itemize}
	\item the preferences of the moral person are \textbf{morally
	authoritative}
	\item the veil of ignorance is Rawl's interpretation of \textbf{moral
	impartiality}
\end{itemize}

\subsection{The veil of ignorance}

\textbf{Information that is excluded:}

\begin{itemize}
	\item the economic, social and cultural development of society
	\item the type of economic and social system
	\item the generation the parties belong to
	\item the social, political and economic position of the parties
	\item their abilities, talents, intelligence, etc.
	\item their conception of the good
	\item their psychological propensities, including their risk attitude
\end{itemize}

\textbf{Information that is included:}

\begin{itemize}
	\item general social and psychological laws
	\item that each person has a conception of the good
	\item that each has a sense of justice
	\item that their society is characterized by the circumstances of
	justice
	\item primary goods
\end{itemize}

\subsection{Primary goods}

Even though you don't know what you want in society, you know there are goods
which are rational to want whatever you want. Thus, you aim at \textbf{
maximizing your bundle of (social) primary goods} for yourself (or for those
your represent).

\textbf{Social primary goods}: rights, liberties and opportunities, power,
income, wealth, (the social bases of) self-respect.

\textbf{Natural primary goods}: health, speed, intelligence, imagination etc.

\begin{itemize}
	\item The distribution of social primary goods can be influenced by
	policy.
	\item Natural primary goods are influenced by the basic structure of
	society, but are not under its direct control.
	\item Primary goods are the means of any other goals which the parties
	have in society.
\end{itemize}

\subsection{The circumstances of justice}

\textbf{You know that the circumstances of justice apply to your society.}

\begin{itemize}
	\item People are physically close, they are roughly equal, and their
	psychological and intellectual abilities are finite.
	\item Goods are only moderately available, but not extremely scarce,
	thus there is competition (\textbf{conflict of interests}):
	\begin{itemize}
		\item if there was abundance, there would be no conflict of
		interests
		\item if there was extreme scarcity, there would be no exchange
	\end{itemize}
	\item People are only moderately other-regarding, thus they need to
	coordinate their actions (\textbf{common interests}):
	\begin{itemize}
		\item if they were too altruistic, there would be no need for
		justice
		\item if they were too egoistic, no constraints could be
		maintained among them
	\end{itemize}
\end{itemize}

\subsection{Constraints on the concept of right}

\textbf{Exclude those conceptions of justice which are in conflict with the
constraints on the concept of right}

\textbf{Generality}: The principles cannot contain proper names, concrete
descriptions, etc., only general properties and relations.

\textbf{Universality}: The principles must apply to all moral persons.

\textbf{Publicity}: The principles must be intelligible and accessible to all,
and they must be commonly known such that everyone can assess and accept them.

\textbf{Ordering}: The principles must be able to evaluate competing claims.

\textbf{Finality}: The principles provide the basis for decisions which are
final; there is no "higher point of view".

\subsection{The maximin rule}

Note that you can be in any position in society, thus you have to select a
conception of justice knowing that you might turn out to be the least
advantaged member of society. To ensure that you have an adequate amount of
primary goods even if you are the least advantaged member of society, choose
the conception of justice using the \textbf{maximin rule}.

If $g_i$ is the index of primary goods of person $i$ in a society with persons
$N = \{1, ..., n\}$, then a social state $x$ is at least as good (or just)
as $y$ iff:

$min(g_1(x), g_2(x), ..., g_n(x)) \geq min(g_1(y), g_2(y), ..., g_n(y))$

\subsection{Utilitatianism}

\subsubsection{Classical (total) utilitarianism}

If $u_i$ is the utility of person $i$ in a society with persons
$N = \{1, ..., n\}$, then a social state $x$ is at least as good (or just) as
$y$ iff:

$\sum_{i=1}^{n} u_i(x) \geq \sum_{i=1}^{n} u_i(y)$

\subsubsection{Modern (average) utilitarianism}

If $u_i$ is the utility of person $i$ in a society with persons
$N = \{1, ..., n\}$, then a social state $x$ is at least as good (or just) as
$y$ iff:

$\frac{1}{n} \sum_{i=1}^{n} u_i(x) \geq \frac{1}{n} \sum_{i=1}^{n} u_i(y)$

\subsection{Justice as fairness}

In the choice between \textbf{utilitarianism} and \textbf{justice as fairness},
choose the latter, since the maximin rule leads to a conception of justice
which does not permit the sacrifice of any individual for the community's
benefit and distributes resources so as to benefit everyone, and in particular
the least advantaged -- thereby you maximize your minimum prospects.
(Utilitarianism violates the \textbf{separatedness of persons}.)

\textbf{Reasons against the principle of utility}:

\begin{enumerate}
	\item There is no basis for probability assignments for the outcomes.
	\item There is no special reason for trying to obtain more than the
	minimum.
	\item The alternative principles have very bad possible outcomes.
\end{enumerate}

\subsection{The principles of justice}

\textbf{(I) First Principle} Each person has an equal righ to the most
extensive schemes of basic liberties that is compatible with a similar scheme
of liberties for others.

\textbf{(II) Second principle} Social and economic inequalities are to be
arranged so that they are \textbf{(a)} reasonably expected to be to everyone's
advantage; \textbf{(b)} attached to positions and offices open to all.

\begin{itemize}
	\item \textbf{(I)} is also known as the \textbf{Principle of Liberty}
	\item \textbf{(II/b)} is known as the \textbf{Principle of Fair
	Equality of Opportunity}
	\item \textbf{(II/a)} is known as the \textbf{Difference principle}
\end{itemize}

\textbf{Priority rules}

Ordering:
\begin{enumerate}
	\item Principle of Liberty
	\item Principle of Fair Equality of Opportunity
	\item Difference Principle
\end{enumerate}

\textbf{The division of labor between the principles}
\begin{itemize}
	\item The Principle of Liberty provides the grounds of political
	equality
	\item The principle of Fair Equality of Opportunity corrects the
	differences of social contingencies
	\begin{itemize}
		\item Fair Equality of Opportunity is more demanding than the
		Formal Equality of Opportunity
	\end{itemize}
	\item The Difference Principle corrects disadvantages caused by
	differences in natural abilities
\end{itemize}

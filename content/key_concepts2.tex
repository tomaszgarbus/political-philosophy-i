\section{Key concepts 2}

\begin{itemize}
	\item \textbf{state of nature}: the pre-political state characterized
	 by no social contract, no property laws etc. In Hobbes' state of
	 nature, life is solitary, poor, nasty, brutish and short.
	\item \textbf{doctrine of equality}: differences between people are
	 insufficient to lead to power imbalances. Consistent with Darwin's
	 self-domestication hypothesis
	\item \textbf{competition, diffidence, desire for glory}: the 3
	 principal cause of quarrel. Competition arises from scarcity.
	 Diffidence is a feeling of insecurity about the future, since noone
	 is able to defend themselves with certainty. The desire for glory is
	 to increase one's security by developing a reputation for strength.
	\item \textbf{laws of nature}:
	 \begin{enumerate}
	 	\item everyone should strive for peace as long as it is
		 obtainable
		\item for the sake of peace, one should lay down their rights
		 and be satisfied with as only much liberty as he would allow
		 be exercised against him
		\item men perform their covenants
	 \end{enumerate}
	\item \textbf{Prisoner’s dilemma}: a game in which defection is always
	 better for each player, regardless what other players do, but
	 if everyone cooperates it gives better outcomes for everyone than if
	 everyone defects. Related to Braess paradox and tragedy of the
	 commons.
	\item \textbf{Hobbes’ conception of justice}: justice arises from the
	 third law of nature, because injustice is just non-compliance to
	 the covenants or to social contract.
	\item \textbf{the Sovereign (sovereignty)}: a person, or group of
	 persons, to which everyone alienates their rights
	\item \textbf{alienation social contract}: it is not possible to
	 revoke the contract. People alienate all their rights, except the
	 right to self-defense which cannot be revoked.
	\item \textbf{bootstrapping problem}: how can social contract arise
	 from the state of nature in which nobody can be trusted to comply with
	 contracts or covenants
	\item \textbf{fear and liberty}: for Hobbes there is no violation of
	 liberty if citizens act only out of fear of the Sovereign
	\item \textbf{the problem of rebellion}: the right of self-defense is
	 not given up, so people can rebel against the Sovereign if they feel
	 threatened by it. Therefore people do not really alienate their
	 rights.
\end{itemize}

\section{Key concepts 5}
\begin{itemize}
	\item \textbf{the individualism objection}: liberalism leads to
	 individualism, atomism and the destruction of traditional social
	 structures
	\item \textbf{Liberty principle (harm principle)}: the only purpose for
	 exercising power against libery of any member of society is to prevent
	 harm for others
	\item \textbf{paternalism}: soft paternalism is interference with a
	 person's libery for the sake of that person's good when the person is
	 irrational, uninformed or incompetent in some way. Hard paternalism is
	 interference with a person's liberty for the sake of that person's
	 good when the person is fully informed and competent.
	\item \textbf{utilitarianism}: utility is the ultimate appeal on all
	 ethical questions
	\item \textbf{utility}: for Mill it consists in happiness and happiness
	 consists in pleasure
	\item \textbf{classical (total) and average (modern) utilitarianism}:
	 classical sums up all utilities of all members of societies, modern
	 averages them.
	\item \textbf{Mill’s theory of well-being and happiness}: at the center
	 of Mill's conception of well-being is the capacity for
	 self-development according to one's own direction and plan for life
	\item \textbf{higher and lower pleasures}: according to Mill the
	 principle of utility should recognize the fact that pleasures differ
	 not only in quantity but also in quality
	\item \textbf{competent judges test}: only someone who experienced both
	 higher and lower pleasure can reliably say which one is better.
	\item \textbf{Socrates and the fool example}: it's better to be a sad
	 Socrates than a happy fool, because then you know both sides of the
	 question, a fool only knows his own side.
	\item \textbf{indirect pursuit of happiness}: only those are happy who
	 have their minds fixed on some object other than their own happiness,
	 for example on the happiness of others, on the improvement of mankind,
	 on some art or pursuit
	\item \textbf{self-development (humans as progressive beings)}: social
	 institutions should enable self-development and thereby human
	 flourishing
	\item \textbf{experiments in living}: the free development of
	 individuality
	\item \textbf{the argument for liberalism}:
	 \begin{enumerate}
	 	\item the best institutions are those under which people are
		 happiest
		\item a precondition of happiness is self-development
		\item self-development requires free individual experimentation
		 in living
		\item the states that best permit and promote free individual
		 experimentation are liberal states
		\item therefore people will be happiest under liberal states
		\item therefore liberal states provide best institutions
	 \end{enumerate}
\end{itemize}

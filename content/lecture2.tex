%%%%%%%%%%%%%%%%%%%%%%%%%%%%%%%%%%%%%%%%%%%%%%%%%%%%%%%%%%%%%%%%%%%%%%%%%%%%%%%%
\section{Lecture 2: Hobbes -- Social Contract Theory}

Thomas Hobbes (1588--1679):

\begin{itemize}
    \item 1603-1608 educated at Oxford, speaks Greek, Latin, French, Italian
    \item 1608 tutor to the Cavendish family and the Prince of Wales, Charles II
    \item "Grand Tours" of Europe, meets Galileo, Descartes, Francis Bacon
    \item c1640 \textbf{Elements of Law} (circulated)
    \item 1640-1655 flees and lives in Paris
    \item 1651 \textbf{Leviathan}
    \item Leviathan offends the French church; returns to England
\end{itemize}

\begin{itemize}
    \item 1962 Charles I dismisses Parliament; without it, he cannot raise
    revenue
    \item 1634-1641 Charles I introduces "ship money", a little-used form of
    tax for national emergencies
    \item Charles I appeals to the divine right of kings (rulers are
    accountable only to God) and that
    \begin{itemize}
        \item Rulers must have the means to defend the state
        \item Only the ruler is entitled to judge if the state is threatened
    \end{itemize}
    \item \textbf{Elements of Law} lends support to this position
\end{itemize}

\begin{itemize}
    \item 1640 Parliament is summoned and outlaws taxation without Parliament's
    approval, Hobbes has to flee to Paris
    \item 1642-1646, 1648-1651 English Civil Wars
    \item \textbf{Leviathan} offends Royalists for its suggestion that
    subjects can abandon a ruler who cannot protect them, Hobbes has to flee to
    England
    \item 1660 Restoration: Charles II protects and supports Hobbes
    \item 1666 Parliament prepares bill against atheism
    \item Hobbes burns his papers
\end{itemize}

Leviathan (1651)

\begin{itemize}
    \item Leviathan: sea monster in the Old Testament
    \item \textit{There is no power on earth to be compared to him} (Job 41)
    \item Book cover
    \begin{itemize}
        \item Sword, castle, crown, cannon, weapons, battle -- \textbf{The 
        power of the State}
        \item Crosier, church, mitre, logic, court, excommunication -- \textbf{
        The power of the Church}
    \end{itemize}
\end{itemize}

\begin{figure}[H]
    \centering
    \begin{tabular}{p{0.45\linewidth} | p{0.45\linewidth}}
        \textbf{Aristotelian tradition} & \textbf{Hobbesian turn} \\
        \hline
        Objective moral good & Good is the object of desire \\
        Actions explained with reference to good & Actions explained with
        reference to self-interest \\
        People are naturally unequal & No significant inequalities between
        people \\
        People are inherently social beings & Methodological individualism \\
        Natural rights are granted by God & Natural rights are precepts of 
        reason (rationality) \\
    \end{tabular}
\end{figure}

\textbf{The social contract argument}

\begin{itemize}
    \item People are characterized by traits \textit{T}
    \item People live in conditions \textit{C}
    \item People with traits \textit{T} in conditions \textit{C} behave in
    ways \textit{B}
    \item People behaving in \textit{B} ways are in state \textit{S}
    \item People could avoid the disadvantages of \textit{S} if they all
    complied with norms \textit{N}
    \item Compliance with norms \textit{N} can only be achieved by
    arrangement \textit{A}
\end{itemize}

\textbf{The doctrine of equality}

Differences between people are insufficient to lead to spontaneous power
imbalances.

Darwin's self-domestication hypothesis is that in every ape group or species
males are much stronger then females and some males are stronger than others.
Humans are an exception, because all males are similarly strong.
Self-domestication hypothesis says that we stopped the "alpha-males" from
reproducing. That's how domestication is done in wolves, by killing (or not
allowing to reproduce) the aggressive and strong ones.

If self-domestication hypothesis is true, than the doctrine of equality is
not natural, we made ourselves that way.

\textit{So that in the nature of man, we find three principal causes of
quarrel. First: competition; secondly: diffidence; thirdly, glory.}

\begin{itemize}
    \item People are motivated by self-preservation and self-interest
    \item When there is scarcity, there is competition, especially given
    equality
    \item Diffidence is a feeling of insecurity about the future, since no
    one is able to defend themselves with certainty
    \item The desire for glory is to increase one's security by developing
    a reputation for strength
    \item In the state of nature, life is \textit{solitary, poor, nasty,
    brutish, and short}.
\end{itemize}

\textbf{The prisoner's dilemma}

\begin{itemize}
    \item Defection is always better for each player than cooperation.
    \item If defection is to attack, it's "war of all against all".
\end{itemize}

\begin{figure}[H]
    \centering
    \begin{tabular}{|c|c|c|}
    \hline
    & Cooperate & Defect \\
    \hline
    Cooperate & 2,2 & 0,3 \\
    \hline
    Defect & 3,0 & 1,1 \\
    \hline
\end{tabular}

\end{figure}
Defection is the \textbf{dominant strategy}\footnote{
    A dominant strategy in game theory is a strategy that always provides a
    better outcome for a player, regardless of what the other players do. This
    means that if a player has a dominant strategy, they will always choose it
    because it maximizes their payoff in every possible scenario.
}.

There is only one equilibrium: (attack, attack). But each player would be
better off if they could cooperate and reach (cooperate, cooperate).

\subsection{The laws of nature}

\textbf{First law of nature}

\textit{Every man ought to endeavour peace, as far as he has hope of
obtaining it; and when he cannot obtain it, that he may seek and use all helps
and advantages of war.}

\textbf{Second law of nature}

\textit{A man be willing, when others are so too, as far forth as for peace
and defense of himself he shall think it necessary, to lay down this right
to all things; and be contended with so much liberty against other men as he
would allow other men against himself.}

\textbf{...in the state of nature}

\begin{itemize}
    \item The state of nature is a pre-political state, in which justice and
    injustice does not arise
    \item But there is natural law: the basic natural right of \textbf{
    self-preservation} (self-defense)
    \item Everyone has the right to everything to secure their natural right
    \item No covenants can be made, since no one can be assured of the other
    party's compliance
\end{itemize}

\textbf{...and human rationality}

\begin{itemize}
    \item Reason can discover the laws of nature: Hobbes calls them
    \textbf{theorems} (contrast geometry!)
    \item A law of nature is a precept or rule of \textbf{rationality}
    \item A law of nature formulates a means to securing the natural right of
    self-preservation
    \item Laws of nature are \textbf{conditional}
\end{itemize}

\textbf{Third law of nature}

\textit{"Men perform their covenants made"} (XV.1), for \textit{covenants,
without the sword, are but words, and of no strength to secure a man at all}
(XVII.2), and \textit{there must be some coercive power to compel men
equally to the performance of their covenants} (XV.3).

\begin{itemize}
    \item A law of nature can serve as a constraint only if others follow it.
    \item Compliance is assured only when there is a greater force that can
    ensure it.
    \item Covenants create obligations even (or especially) when they are made
    out of fear.
    \item The laws don't limit people's rights: they remove a constraint (the
    threat posed by others).
    \item Justice arises from the third law of nature, because injustice is
    just non-compliance.
\end{itemize}

\subsection{The creation of political authority (the Sovereign)}

\begin{itemize}
    \item Each person mutually agrees that they transfer their rights to
    the sovereign.
    \item In modern terms, people \textbf{alienate} their rights.
    \item People as a collective authorize someone(s) with political
    authority.
    \item The authorization itself is \textbf{non-contractual}: the Sovereign
    is not a party to the social contract (the people contract with one
    another).
    \item Therefore, the political authority of the Sovereign does not come
    from the contract.
    \item Therefore, it is not possible to void or revoke the contract.
\end{itemize}

The rights of the Sovereign:

\begin{itemize}
    \item The right to determine the means of peace and defense
    \item The right to judge views which may be useful or harmful for peace
    \item The right to make peace and war with other nations
    \item ...
\end{itemize}
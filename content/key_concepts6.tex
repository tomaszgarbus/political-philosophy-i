\section{Key concepts 6}

\begin{itemize}
	\item \textbf{natural law, natural rights, civil rights}: natural
	 rights are those which have source not in politics but in human
	 nature (or God). Natural law tradition says that human rights are the
	 natural rights. Civil rights are those which apperain to main right of
	 being a member of society.
	\item \textbf{moral rights and human rights}: moral rights, unlike
	 human rights, may not be codified in law
	\item \textbf{features of human rights}:
	\begin{itemize}
		\item \textbf{unconditional}
		\item \textbf{inalienable}
		\item \textbf{universal}
		\item \textbf{prepolitical}
		\item \textbf{institutional}
	\end{itemize}
	\item \textbf{human rights and obligations}: the obligation holder for
	 human rights is political authority: violations of human rights are
	 acts committed in an official capacity
	\item \textbf{human rights and political authority}: you can only be
	 autonomous moral subject (and therefore the political authority can
	 demand obedience from you) if you have basic rights, only then can it
	 present itself as morally justified towards its subject
\end{itemize}

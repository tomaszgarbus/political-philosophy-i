\section{Key concepts 1}

\begin{itemize}
	\item \textbf{power}: X has power over Y if X can compel Y to perform
	 some action. X's power over Y gives Y a reason to obey but it does not
	 create an obligation
	\item \textbf{authority}: X has authority over Y if X occupies a social
	 role from which they can provide reasons for Y to act in certain ways.
	 X's authority gives Y a reason to obey but it does not create an
	 obligation
	\item \textbf{political authority}: X has political authority over Y
	 iff the fact hat X requires Y to perform some action p gives Y a
	 readon to perform p, regardless of what p is and this reason purports
	 to override all (or nearly all) reasons Y may have not to perform p
	\item \textbf{preemptive (or exclusionary) reasons}: commands issued by
	 political authority provide reasons that \textit{preempt} or
	 \textbf{override} other reasons. Political authority requires
	 surrender of judgement
	\item \textbf{natural subordination theory}: some creatures
	 instinctively submit to others whose nature makes them fit for rule,
	 dominance and power. Natural subordination theories can be further
	 divided between \textit{natural roles} argument (e.g. Aristotle's
	 natural domination of men over women) and \textit{consequentialist}
	 argument (e.g. Mill's enlightened colonialism)
	\item \textbf{divine authority theory}: ruler's political authority
	 comes from God, either the ruler is (related to) God (Egyptian
	 pharaos, Dalai Lamas, some Roman emperors, Japanese emperors) or was
	 granted the authority by God (Adam was given authority to rule the
	 Earth by God and kings are his first-in-line descendants)
	\item \textbf{natural duty accounts of political obligation}: there is
	 a general non-voluntary obligation to maintain and promote just
	 institutions
	\item \textbf{associative accounts of political obligation}: there are
	 special, non-voluntary obligations towards one's own political
	 institutions created by social roles or identities (e.g. obligation
	 from gratitude)
	\item \textbf{transactional accounts of political obligation}: there
	 are special, non-voluntary obligations towards political institutions
	 based on the requirement of reciprocity
	\item \textbf{social contract theories of political authority}:
	 political obligations arise from voluntary acts (consent) either
	 explicitly or implicitly
	\item \textbf{political legitimacy}: justified entitlement to rule.
	 Two different views how it relates to political authority: 1)
	 political authority presupposes legitimacy (only legitimate
	 governments have political authority, illegitimate governments only
	 have power), 2) the two are distinct
	\item \textbf{philosophical anarchism}: there are no preemptive
	 reasons and therefore no political authority. The authority of the
	 state is not unique, it may have authority but not political
	 authority
\end{itemize}

\section{Reading assignment: John Stuart Mill -- On Liberty}

\subsection{Introduction}

\begin{itemize}
	\item concerned with Civil, or Social, Liberty: the nature and limits
	of power that can be legitimately exercised over an individual by the
	society
	\item ancient governments (except Greek democracy) always had an
	antagonism between the ruler and the ruled
	\item \textit{To prevent the weaker members of the community
	from being preyed upon by innumerable vultures, it
	was needful that there should be an animal of prey
	stronger than the rest, commissioned to keep them
	down.}
	\item but in reality the rulers usually preyed on the people
	\item the attempt to limit the rulers' power was first called liberty
	\item the attempts at liberty were done in two ways:
	\begin{itemize}
		\item by obtaining a recognition of certain political
		immunities, called political liberties or rights
		\item by establishment of constitutional checks
	\end{itemize}
	\item over time, people learned to prefer to delegate power to the
	government, not alienate it
	\item in the next step, people wanted the rulers to be identified with
	the people; that their will should be the interest of the nation
	\item protection against the tyrrany of the magistrate is not enough.
	Hamrful laws can be also passed by the public opinion.
	\item \textit{There is a limit to the legitimate
	interference of collective opinion with individual independence; and
	to find that limit, and maintain it against encroachment, is as
	indispensable to a good condition of human affairs, as protection
	against political despotism.}
\end{itemize}

The essay's main principle: \textit{that the sole end for which mankind are
warranted, individually or collectively in interfering with the liberty of
action of any of their number, is self-protection. That the only purpose for
which power can be rightfully exercised over any member of a civilized
community, against his will, is to prevent harm to others. His own good,
either physical or moral, is not a sufficient warrant.}


\section{Key concepts 3}

\begin{itemize}
	\item \textbf{Filmer’s theory}: \textit{Partiarcha, or the Natural
	 Power of Kings}. Since noone can dispose over their life (suicide is
	 sin) but rules have the rights to dispose lives of people, political
	 authority cannot come from people. Hence authorization must come from
	 God (through Adam, Noah and their descendants).
	\item \textbf{the law of nature}: (Locke) preservation of all mankind,
	 everyone is created by God and everyone is equal, everyone is bound
	 by self-preservation and should recognize everyone's self-preservation
	 as equally important. Execution of this law is everyone's duty in the
	 state of nature.
	\item \textbf{state of war}: everyone has the right to defend
	 themselves. A threat -- intention of harming another -- puts the
	 parties in the state of war. Slavery is also the state of war.
	\item \textbf{the problem of irrationality and the assurance problem}:
	 in transgressing the law of nature, the offender declares himself to
	 be living by another rule than that of reason and common equity
	\item \textbf{natural vs civil liberty}: natural liberty is to be under
	 no restraint other than the law of nature. Civil liberty consists in
	 rule of law that applies to everyone equally, laws that are created by
	 representative government, liberty in those things which are not
	 governed by law, freedom from the arbitrary will of another man.
	\item \textbf{civil society vs commonwealth}: in civil society, people
	 put their powers together, in commonwealth they place their powers in
	 the government (by agency or alienation, for Locke it's agency).
	\item \textbf{alienation social contract vs agency social contract}:
	 in alienation (Hobbes) people hand over their rights to the sovereign
	 and cannot get them back, in agency (Locke), the sovereign is there to
	 \textit{serve} the people. In alienation, there is no limit on the
	 political authority of the Sovereign. In agency, people put these
	 limits on the government.
	\item \textbf{actual vs tacit consent}: explicit vs implicit consent
	\item \textbf{rule of law}: applies equally to all, regulates the
	 liberty of man under government
	\item \textbf{self-ownership}: Locke's starting point to the theory of
	 property. Everyone has property rights in their own person (their
	 body, mind and labor). The right to life and liberty is therefor a
	 \textit{property right}. The source of property rights is not the
	 Sovereign. Property rights are prior to political society for Locke.
	\item \textbf{mixing labor theory}: the original claim to property is
	 based on addig labor to resources. The justification of property
	 comes from the additional value created by labor.
	\item \textbf{the Lockean proviso}: the appropriation of property is
	 justified iff "enough and just as good" is left. That is, a person
	 gets property rights only over resources which she actually uses.
\end{itemize}

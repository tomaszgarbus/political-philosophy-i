\section{Lecture 4: Popular Sovereignty -- Rousseau}

\textbf{Jean-Jacques Rousseau} (1712-1778)

\begin{itemize}
	\item born in Geneva ("a free state")
	\item 1750 wins the prize of Academy of Dijon with the
	\textbf{Discourse on the Science and Arts (First Discourse)}
	\item 1755 \textbf{Discourse on the Origins of Inequality (Second
	Discourse)}
	\item 1762 Emilie and The Social Contract -- both condemned in Geneve
	and he has to flee
	\item 1765 he tries to settle in England
	\item 1767 returns to France under false name
	\item 1768 (illegally) marries his companion
	\item 1770 allowed to return to Paris but not to publish
\end{itemize}

\subsection{The state of nature}

\textbf{Rousseau's question}: How did humans become social beings? How did
political communities (conventions) arise?


\begin{itemize}
	\item The only natural form of association between humans is
	\textbf{family}
	\item In the state of nature, people are \textbf{self-sufficient}:
	there is no community between them
	\item Self-sufficient people are free
\end{itemize}

\textbf{Natural rights}

Norms of reason that specify the general preconditions for human existence
and survival. Humans can recognize natural rights (they conform to human
nature).

\textbf{Political rights}

Particular social rules, laws and relations, arising from power relations.

\subsection{Sources of natural rights}

\textbf{Self-preservation}: the propensity to pursue one's self-interest

\textbf{Pity} (empathy): the ability to refrain from harming others

\begin{itemize}
	\item pre-social, self-sufficient humans can spontaneously recognize
	and follow natural rights
	\item social, psychologically independent humans need laws to govern
	their relations
	\item when the state of nature is left behind, human psychology changes
	\item the function of government and the rule of law is to restore
	justice that prevailed among self-sufficient humans
	\item political society is an arrangement put in place of the state of
	nature
\end{itemize}

Hence the opening sentence of The Social Contract: "taking men as they are,"
and "the laws as they can be."

\subsection{Modelling the state of nature}

\textbf{Stag hunt game}

\begin{figure}[H]
    \centering
    \begin{tabular}{|c|c|c|}
    \hline
    & Cooperate & Defect \\
    \hline
    Cooperate & 5,5 & 0,4 \\
    \hline
    Defect & 4,0 & 2,2 \\
    \hline
\end{tabular}
\end{figure}

\begin{itemize}
	\item a version of assurance game
	\item each player attaches $p$ to the other cooperating (\textbf{
	trust})
	\item they can expect $p$ by cooperating, $4p+2(1-p)$ by defecting:
	$5p > 4p + 2(p-1) \implies p > 2 - 2p \implies 3p > 2
	\implies p > \frac{2}{3}$
	\item that's a lot of trust!
	\item promises convey no information on intentions
\end{itemize}

\subsection{Psychological change: the state of nature}

In the state of nature, humans are characterized by:

\begin{itemize}
	\item self-love (self-concern)
	\item pity (empathy or compassion)
	\item perfectibility (psychological adaptability)
\end{itemize}

The consequences of this psychological transformation:

\begin{itemize}
	\item competition
	\item comparing oneself with others
	\item hatred, bitterness
	\item desire for power
\end{itemize}

As humans emerge from the state of nature (through more frequent human contact)
their self-love develops into \textbf{amour propre}: a form of love of self
that is the function of one's esteem by others, determined by pride, envy,
jelaousy, greed...

\subsection{Psychological change: civil society}

\textbf{Psychological change}

\begin{itemize}
	\item in the state of nature people are equal because they are
	independent from each other (no power relations arise from natural
	inequalities)
	\item in civil society people are free only if their equality is
	constantly reinforced by institutions
	\item pity is no longer important as motivational force, it is
	replaced by \textbf{reciprocity} and \textbf{amour propre}
\end{itemize}

\textbf{Moral change}

\begin{itemize}
	\item since people arise from the state of nature in unequal condition
	they have to be \textbf{made equal by convention} to be able to take
	part in the common life
	\item natural inequalities are replaced by \textbf{moral equality} and
	\textbf{equality before the law}
\end{itemize}

\subsection{The general will}

\textbf{The social contract}

\textit{
"A form of association which will defend and protect with the whole common
force the person and goods of each associate, and in which each, while uniting
himself with all, may still obey himself alone, and remain as free as before."
}

The \textbf{total alienation} of rights and powers to become part of the
Sovereign:

\textit{
"Each of us puts his person and all his power in common under the supreme
direction of the general will, and, in our corporate capacity, we receive each
member as an indivisible part of the whole."
}

\textit{
"Each individual, as a man, may have a particular will contrary or dissimilar
to the general will which he has as a citizen"
}

\textit{
Whoever refuses to obey the general will shall be compelled to do so by the
whole body.
\textbf{This means nothing less than that he will be forced to be free.}
}


\begin{itemize}
	\item The general will is always right: it "wills" the general good
	-- the common will of citizens concerned not with pursuing their
	own interests but the well-being of society.
	\item The general will and the \textbf{will of all} may be incongruent
	since people may be irrational and fail to recognize the general will.
	\item A legitimate political authority will defend people and goods
	"with the full common force" while each person "uniting with all"
	"obeys only himself and remains as free as before".
	\item There is no need for checks and balances on popular sovereignty
	but it must be able to enforce the obedience of those who do not
	obey it.
	\item Since everyone takes part in the institutional framework of the
	general will, \textbf{obeying the law is to obey ourselves}.
\end{itemize}

\subsection{Making sense of the general will}

\textbf{Building a dam}

\begin{figure}[H]
	\begin{tabular}{|p{0.3\linewidth}|p{0.3\linewidth}|p{0.3\linewidth}|}
	\hline
	& River goes left & River goes right \\
	\hline
	Left-side farmers & 6 & 6 \\
	\hline
	Right-side farmers & 6 & 6 \\
	\hline
	Aggregate & 6 & 6 \\
	\hline
	\end{tabular}
\end{figure}

\textbf{Letting the river flood}

\begin{figure}[H]
	\begin{tabular}{|p{0.3\linewidth}|p{0.3\linewidth}|p{0.3\linewidth}|}
	\hline
	& River goes left & River goes right \\
	\hline
	Left-side farmers & 0 & 10 \\
	\hline
	Right-side farmers & 10 & 0 \\
	\hline
	Aggregate & 10 & 10 \\
	\hline
	\end{tabular}
\end{figure}

\begin{itemize}
	\item The will of all is to let the river flood
	\item The general will is to build a dam, even though no one wills it
\end{itemize}

\subsection{Is the idea of general will coherent?}

\textbf{The Condorcet paradox}

\begin{figure}[H]
	\begin{tabular}{p{0.3\linewidth}p{0.3\linewidth}p{0.3\linewidth}|}
	Voter 1 & Voter 2 & Voter 3 \\
	\hline
	A & B & C \\
	B & C & A \\
	C & A & B \\
	\end{tabular}
\end{figure}

Using majority rule in pairwise comparisons:
$A \succ B \succ C \succ A$

\textbf{Agenda setting}: the order in which alternatives are introduced can
determine the outcome

\subsection{The relation between sovereignty and government}

\textbf{Forms of government}

Popular sovereignty (the general will) is concerned with general matters
(making laws). But what about implementing laws and everyday
administratoin?


\begin{itemize}
	\item \textbf{Direct democracy}: danger of interference from popular
	sovereignty 
	\item \textbf{Absolute monarchy}: danger of interference from the
	government
	\item \textbf{Aristocracy}
	\begin{itemize}
		\item \textbf{Natural}: not suitable for modern political
		communities (perhaps for primitive people)
		\item \textbf{Hereditary}: the worst form of government
		\item \textbf{Elective aristocracy}: the best form of
		government, combining consent and wisdom (we would call this
		today \textit{elitism} or \textit{epistocracy})
	\end{itemize}
\end{itemize}

\subsection{Rousseau's totalitarianism}

\textbf{Hobbes' problem}

People cannot alienate their right of self-defence, therefore they have the
right to defend themselves against the Sovereign (and not obey the "general
will").

\textbf{Rousseau's solution}

When entering civil society, a \textbf{second psychological transformation}
takes place, and people's will never conflicts with the general will.

\begin{itemize}
	\item If it does, the transformation has not been complete and people
	suffer from "fake consciousness".
	\item People's "real" interests and freedom is in the following the
	general will, even if they need to be coerced.
	\item This more than foreshadows totalitarian dictatorships
	(fascism, communism, ...)
\end{itemize}

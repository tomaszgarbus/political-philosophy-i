\section{John Rawls: A Theory of Justice}

\subsection{Justice as fairness}

\begin{itemize}
	\item sketch of main ideas of theory of justice
	\item classical utilitarian and intuitionist conceptions of justice
	and consider differences between them and justice as fairness
	\item primary subject of justice: the basic structure of society
	\item aims to develop a theory of justice that is alternative to
	dominant doctrines
\end{itemize}

\subsection{The role of justice}

\begin{itemize}
	\item intuitive conviction of primacy of justice:
	\begin{itemize}
		\item the main goal of laws and institutions
		\item welfare of society can't override a person's
		inviolability founded on justice
		\item justice doesn't allow sacrificing few for the good of
		many
		\item injustice is tolerable only when it serves to avoid
		greater injustice
		\item truth and justice are uncompromising
	\end{itemize}
	\item the role of the principles of justice
	\begin{itemize}
		\item suppose that a society is a more or less self-sufficient
		association of persons who live under some social contract
		\item suppose further that the social contract is designed for
		the good of those taking part in it
		\item then, the society is characterized by both conflict and
		identity of interests
		\item there is identity of interests since cooperation benefits
		all
		\item there is conflict because people care who gets how many
		benefits
		\item principles of social justice determine the norms who
		benefits how much from the social cooperation
	\end{itemize}
	\item society is well-ordered when it if regulated by public
	conception of justice and it benefits its members by design
	\begin{itemize}
		\item in such society
		\begin{itemize}
			\item everyone accepts same principles of justice
			\item social institutions satisfy these principles
		\end{itemize}
		\item in such society, if one member demands unjustly much from
		another then the principles of justice create a common ground
		to regulate such demands
		\item in such society, the general desire for justice limits
		the pursuit of other ends
		\item existing societies are rarely well-ordered
		\item what is just and unjust is usually in dispute
		\item still, people recognize the need for a set of principles
		of basic rights and duties
		\item people with different conceptions of justice can agree
		that institutions are just when they act on principles and not
		arbitrarily
		\item distinction between \textbf{the concept of justice}
		(to base institutions on a set of principles) and \textbf{the
		conceptions of justice} (specific sets of principles)
	\end{itemize}
	\item some degree of agreement in conceptions of justice is needed for
	a human community
	\item other problems are coordination, efficiency, stability:
	\begin{itemize}
		\item the plans of individuals must be fitted together
		\item execution of social ends should be done efficiently and
		consistently with justice
		\item the cooperation must be stable, that is based on stable
		principles
	\end{itemize}
	\item lack of justice leads to distrust and resentment, these in turns
	ruins society and human activity
	\item \textit{one conception of justice is preferable to another when
	its broader consequences are more desirable}
\end{itemize}

\subsection{The subject of justice}

\begin{itemize}
	\item \textit{primary subject of justice is the basic structure of
	society}
	\item fundamental rights and duties
	\item division of advantages from social cooperation
	\item social institutions examples: legal protection of freedom of
	thought, competitive markets, private property as means of production,
	monogamous family
	\item deep inequalities, or starting places of different members of
	society
	\item principles of social justice must primarily address these
	inequalities
	\item limits on the scope of inquiry:
	\begin{enumerate}
		\item not concerned with justice between states nor nations
		\item not concerned with general case, i.e. principles
		satisfying all possible scenarios
		\item concerned only with society as a closed system
		independent from other societies
		\item concerned only with well-ordered society, everyone is
		presumed to act justly and uphold just institutions
		\item what would a perfectly just society be like?
		\item consider only strict compliance, not partial
	\end{enumerate}
	\item the concept of the basic structure
	\begin{itemize}
		\item which institutions should be included?
	\end{itemize}
	\item a conception of social justice
	\begin{itemize}
		\item assess the distributive aspects of the basic structure
		of society
		\item social ideal
		\item principles of justice are the most important part
		\item social idea connects them with a conception of society,
		with aims and goals of social cooperation
		\item various conceptions of justice are results of different
		notions of society
	\end{itemize}
	\item \textit{
		any reasonably complete ethical theory must include principles
		for distributive principles for the basic structure of society
	}
	\item \textit{the \textbf{concept of justice} I take to be defined
	by the role
	of its principles in assigning rights and duties and indefining the
	appropriate division of social advantages}
	\item \textit{a \textbf{conception of justice} is an interpretation of
	this role}
	\item is this approach consistent with tradition?
	\begin{itemize}
		\item Aristotle about justice: refraining from \textit{
		pleonexia}\footnote{greed}
	\end{itemize}
\end{itemize}

\subsection{The main idea of the theory of justice}
\begin{itemize}
	\item justice as fairness
	\begin{itemize}
		\item goal: create a social contract theory more general and at
		higher abstraction level than Locke, Rousseau and Kant
		\item social contract should be based on principles that every
		free and rational person would accept
		\item these principles shall regulate all further agreements
		\item participants in social cooperation choose together the
		principles upon which they build the concept/conception (?) of
		justice
		\item just like each person must decide what's good and evil,
		a group of persons must decide together what's just and unjust
		\item the original position of equality corresponds to the
		state of nature in the traditional theory of the social
		contract
		\item the essence of justice as fairness: in th state of nature
		noone has any leverage for rigging the system in his favour so
		the agreed-on social contract must be fair
		\item it does not mean that justice and fairness are the same
		\item a social situation is just if we would have consented to
		it in a scenario of going from the state of nature through
		subsequent steps of development of social contract
		\item treats parties in the initial situation as rational and
		mutually disinterested
	\end{itemize}
	\item principle of utility -- would it be accepted by people in the
	position of equality?
	\item if we understand the principle of utility (like Rawls does) as
	the algebraic sum of everyone's good, than no, becauce people want to
	protect their interests and won't sacrifice them for general sum of
	advantages
	\item so the principle of utility since incompatible with social
	cooperation among equals for mutual advantage
	\item notion of reciprocity is implicity in the notion of a 
	well-ordered society
	\item persons in the initial situation of equality would rather choose
	2 different principles:
	\begin{enumerate}
		\item equality in assignment of rights and duties
		\item social and economic inequalities, for example
		inequalities of wealth and authority, are only just if they
		compensate everyone
	\end{enumerate}
	\item justice as fairness consists of two parts:
	\begin{enumerate}
		\item an interpretation of the initial situation and problem
		of choice
		\item a set of principles which according to the theory would
		be agreed to
	\end{enumerate}
	\item justice as fairness is an example of a contract theory
\end{itemize}

\subsection{The original position and justification}
\begin{itemize}
	\item one conception of justice is more reasonable and more justifiable
	than another if rational persons in the initial situation would choose
	its principles over those for the role of justice
	\item connection between theory of justice and theory of rational
	choice -- which principles would be chosen by rational beings in
	initial situation?
	\item this depends on the interpretation of initial situation
	\item contract approach: collect weak but widely accepted premises into
	more specific conclusions
	\item \textbf{reflective equilibrium}
	\begin{itemize}
		\item we search for description of initial situation, starting
		from both ends
		\item start with shared and weak conditions
		\item derive set of principles from these conditions
		\item if not, look for more premises
		\item if we find friction between conclusions and our
		considered convictions of justice, we can either modify the
		account of the initial situation or revise our existing
		judgements
		\item we repeat back and forth
		\item finally we reach the reflective equilibrium
		\item it is equilibrium because our principles and judgemenents
		coincide
		\item it is reflective because the judgements reflect the
		principles
	\end{itemize}
\end{itemize}

\subsection{Classical utilitarianism}

\begin{itemize}
	\item goal: work out a theory of justice alternative to utilitarian
	thought
	\item contrast between the contract view and utilitarianism
	\item the structure of an ethical theory depends on how it defines
	and connects the two notions:
	\begin{enumerate}
		\item the right
		\item the good
	\end{enumerate}
	\item teleological theories: the good is defined, then the right is
	defined as that which maximizes the good
	\item teleological theories intuitively seem to be rational
	\item teleological theory allows us to judge what things are good
	without considering if they are right -- for example this leads to
	maximizing pleasure
	\item teleological theories depend on how we define the good:
	\begin{itemize}
		\item realization of human excellence: perfectionism
		(Aristotle, Nietzsche)
		\item pleasure: hedonism
		\item happiness: eudaimonism
		\item etc.
	\end{itemize}
	\item Rawls defines it as satisfaction of rational desire
	\item for utilitarian view of justice does not matter how the sum of
	satisfactions is distributed among individuals
\end{itemize}

\subsection{Two principles of justice}

\begin{itemize}
	\item two principles of justice should be agreed to in the original
	position
	\item first formulation:
	\begin{enumerate}
		\item \textit{each person is to have an equal right to the
		most extensive scheme of equal basic liberties compatible with
		a similar scheme of liberties for others}
		\item \textit{social and ecomomic inequalities are to be
		arranged so that they are both (a) reasonably expected to be
		to everyone's advantage and (b) attached to positions and
		offices open to all}
	\end{enumerate}
	\item the following liberties are to be equal by the first principle:
	political liberty (to vote and hold office), freedom of speech and
	assembly, liberty of conscience and freedom of thought, freedom of
	person, right to hold personal property, freedom from arbitrary arrest
	and seizure
	\item injustice is simply inequalities that are not to the benefit of
	all
	\item exchanging between basic liberties and economic and social gains
	is not permissible in this theory (because the basic liberties rule
	is the first one)
\end{itemize}

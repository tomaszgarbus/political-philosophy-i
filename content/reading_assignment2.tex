\section{Reading assignment 2: Hobbes - Leviathan}

\subsection{Of the Natural Condition of Mankind as Concerning Their Felicity
and Misery}

1. People differ in physical strength and mental capabilities but when all
taken together, no man can claim superiority over another.

2. All men think they have bigger mental capabilities than the average. This
is a good sign of equal distribution that everyone is happy with their share.

3. From the equality of ability arises equality of wants. Men become enemies
when they want something which cannot be shared between them.

4. Thus one must increase his power to secure his position and possessions.
Some take pleasure in acquiring more power than necessary.

5. One looks for companions that have equal power. Inequalities between peers
create grief and damage.

6. Three causes of conflict: competition, diffidence, glory.

7. They make made invade (respectively) for: gain, safety, reputation.

8. War is the default state of men. War is not only manifested in fighting
but also in disposition thereto. Peace is assurance of non-conflict.

9. The state of war creates too much uncertainty for the industry or
cultivation of earth to be practiced or developed.

10. Our daily experience confirms the state of war: we travel with companions,
we lock our doors, we lock our chests, we pay great deal of attention to
security.

11. "Savages" in America live to this day in their "brutish" ways because there
is no state or laws to regulate them.

12. Kings are in constant state of (cold) war with one another. But because
they exercise control over their citizens, the citizens are not in war with
one another.

13. Justice and injustice, good and evil, right and wrong, are societal notions
and have no meaning in the state of war between men.

14. Men are inclined to peace by: fear of death, desire of necessary things
for living, hope by their industry to obtain them.

\subsection{Of the First and Second Natural Laws, and of Contracts}

1. The \textbf{Right of nature} (\textit{jus naturale}) is the liberty of each
man to his own life and to defend it.

2. By \textbf{Liberty} we mean lack of external impediments preventing man from
exercising his will (freedom \textit{from}).

3. The \textbf{Law of nature} (\textit{lex naturalis}) is man's obligation to
sustain and defend his life.

4. Every man has obligation to seek peace as long as he has hope of obtaining
it and right to defend him otherwise. These follow respectively from law of
nature and right of nature.

5. In order to create peace we form the second law: every man shall exercise
his liberty in relation to others only as much as he will allow it to be
exercised againt him.

6.

7. Difference between \textbf{renouncing} and \textbf{transferring} right.

8. Rights can be only transferred voluntarily in expectation for some good in
return. For example one cannot lay down their right of resisting assault
because this can only harm them.

9. A \textbf{contract} is a mutual transferring of rights.

11. A contract can be called \textbf{pact} or \textbf{covenant} if it is to be
delivered in the future by one side.

18. When both parties perform their part of the contract in the future, it is
void unless there is no power governing both of them to hold their promises.

19. Civil estate guarantees the execution of the contract.

20. That which cannot hinder man from performing his part of the contract
cannot be admitted as the cause of hindrance.

27. Covenants entered through fear (such as ransom for a prisoner of war or
kidnapping) have to be respected until rendered void by the civil law.

28. A former covenant makes void a later.

29. A covenant not to defend myself from force by force is always void. A man
may covenant \textit{unless I do so, kill me} but he cannot covenant \textit{
unless I do so, I will not resist when you come to kill me}.

31. Two elements of human nature help make sure man keeps his word: fear of 
consequence for breaking one's word and glory or pride in appearing not to
break one's word.

\subsection{Of Other Laws of Nature}

1.

2. \textbf{Injustice} is defined as \textit{not performing the covenant}.
Everything that is not unjust is just.

3. There must be some coercive power to force men to be just.

40. The way, or means, of peace are justice, gratitude, modesty, equity,
mercy.

41. \textit{But yet if we consider the same theorems as delivered in the word
of God that by right commands all things, then are they properly called laws.}

\subsection{Of the Causes, Generation, and Definition of a Commonwealth}

1. The final cause of men in the introduction of restraint upon themselves is
to stop the misery of war.

2.

3. The multitude united in one person (or assembly of persons) is called a
\textbf{commonwealth}, in Latin \textit{civitas}.

14. He that carries that person is called \textbf{sovereign}, is said to have
\textbf{sovereign power} and everyone besides is his \textbf{subject}.

\subsection{Of the Rights of Sovereigns by Institution}

Commonwealth is established once everyone makes covenant with everyone else to
respect the man or assembly given that role. Commonwealth is granted the
\textit{right to present} (represent).

\subsection{Of the Liberty of Subjects}

1. Liberty is defined as absence of external opposition.

3. A man in commonwealth obeys the law for the \textit{fear} of consequences
but he has the \textit{liberty} to fuck around and find out.

4. \textit{Liberty} and \textit{necessity} are consistent because every action
man does willingly proceeds from some cause. The root of all causes are initial
actions of God.

12. If the sovereign commands a man to kill, wound, or in other way harm
himself, he has the right to disobey.

13. One is not obliged to confess a crime he has committed (without assurance
of pardon).

17. To resist the sword of the Commonwealth in defence of another man, guilty
or innocent, no man has liberty.

21. The obligation of subjects to the sovereign lasts only as long as the
sovereign is able to protect them.

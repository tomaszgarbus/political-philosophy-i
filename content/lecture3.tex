%%%%%%%%%%%%%%%%%%%%%%%%%%%%%%%%%%%%%%%%%%%%%%%%%%%%%%%%%%%%%%%%%%%%%%%%%%%%%%%%
\section{Lecture 2 -- continuation}

\subsection{Problems with Hobbes}

\textbf{Bootstrapping}

No contracts can be made or maintained in the state of nature -- but a contract
to create a power that force compliance and makes other contracts possible
is possible.

\textbf{Fear and liberty}

For Hobbes, fear and liberty are not in contradiction: there is no violation of
liberty if citizens act only out of fear of the Sovereign. But then the
Sovereign's rule seems based on power, rather than genuine form of political
authority.

\textbf{Rebellion}

\begin{itemize}
    \item Even once the contract is made, the right to self-defense is not given
    up -- no one can make a contract promising not to defend themselves.
    \item Citizens have an obligation to obey the Sovereign only as long as it
    is able to protect them.
    \item What happens if the subjects cease to believe that the Sovereign can
    protect them (or feel threatened by it)?
    \item Since only they can determine when their preservation is threatened,
    they have the right to determine whether they should obey (or mount a
    rebellion)
    \item Therefore, they do not really \textit{alienate} all their rights.
\end{itemize}

%%%%%%%%%%%%%%%%%%%%%%%%%%%%%%%%%%%%%%%%%%%%%%%%%%%%%%%%%%%%%%%%%%%%%%%%%%%%%%%%

\section{Lecture 3: Locke -- Representative Government}

John Locke (1632-1704)

\begin{itemize}
    \item 1647-1658 educated in London, Oxford
    \item 1667 personal physician of 1st Earl of Shaftesbury
    \item 1672 Lord Shaftesbury becomes Lord Chancellor
    \item 1675-1679 travels in Europe (Two Treatises of Government probably
    written around this time)
    \item 1683 has to flee to the Netherlands
    \item 1683-1688 prepares Two Treatises of Government and Letter Concerning
    Toleration for publication
    \item 1688 Glorious Revolution
    \item 1689 returns to England and published his major works
\end{itemize}

\subsection{Two Treatises of Government (1689)}

\textbf{Ultimate aim}: to justify the idea that subjects can rebel against their
rulers.

\textbf{Further questions}:

\begin{itemize}
    \item What distinguishes authority in the family and in the state?
    \item What is the relation of ruler and subjects?
    \item How are property rights possible?
\end{itemize}

\textbf{Main opponent: Sir Robert Filmer (1588-1653), \textit{Patriarcha, or
the Natural Power of Kings}}. Since no-one can dispose over their life (e.g.
suicide is a sin, one's life is in the hands of God), but rulers have the right
to dispose over the lives of people (subjects are literally the property of
their rulers), political authority cannot come from the people. Hence
authorization must come from God (through Adam, Noah, and their descendants).

\subsection{The state of nature}

If a Swiss person goes to America and meets and Indian, they are in the state of
nature in relation to each other because there is no common civil society
governing them.

Robert Nozick -- \textit{Anarchy, State, and Utopia}

\subsubsection{The law of nature}

\textbf{Preservation of all mankind}

\begin{itemize}
    \item Everyone is create by God and everyone is equal.
    \item Everyone is bound by self-preservation and should mutually recognize
    that everyone's preservation is equally important.
\end{itemize}

People are politically equal; if they are rational, they respect the equality of
others. As long as the law of nature are respected, there is no need for
political authority. In the state of nature, the \textit{execution} of the law
of nature is everyone's duty.

\textit{Every man hath a right to punish the offender, and be executioner of the
Law of Nature.}

\subsubsection{The problem of irrationality}

\textit{In transgressing the law of nature, the offender declares himself to
live by another rule than that of \textbf{reason} and common equity.}

\begin{itemize}
    \item When irrationality is present, the law of nature is not respected
    anymore.
    \item The punishment of violators creates the problem of
    \textit{impartiality}: who can adjudicate between conflicts (that is,
    interpret the law of nature) in an impartial way?
    \item The (cooperative) state of nature becomes a state of war.
\end{itemize}

Example: WWI, no one wanted to go to war, but mutual offenses just escalated.

\subsubsection{Modeling the state of nature}

\textbf{The assurance problem}

\begin{figure}[H]
    \centering
    \begin{tabular}{|c|c|c|}
        \hline
        & Cooperate & Defect \\
        \hline
        Cooperate & 3,3 & 0,2 \\
        \hline
        Defect & 2,0 & 1,1 \\
        \hline
    \end{tabular}
\end{figure}

Difference from PD -- here cooperation is not irrational.

\begin{itemize}
    \item There is no dominant strategy.
    \item Two equilibria are $(3, 3)$ and $(1, 1)$ (state of war).
    \item Problem: \textit{mistakes} -- imperfect rationality may lead to the
    state of war.
\end{itemize}

\subsection{State of war}

Everyone has the right to defend themselves against those who threaten them.
A \textit{threat} -- an intention of harming another -- puts the parties in
state of war; thus, anyone who tries to gain absolute power over others puts
himself in a state of war with regard to them.

\begin{itemize}
    \item Slavery is also the state of war (the "state of war continued")
    between master and slave (this is why slavery is incompatible with civil
    government).
\end{itemize}

\subsection{The social contract}

The social contract is the codification of the law of nature in civil laws and
institutional structures.

The end of the political authority is to solve assurance problems by creating
\textit{laws}, resolving conflicts \textit{impartially}, and enforcing the laws
and impose punishments.

\subsection{Political authority}

\textbf{Natural liberty}: \textit{To be under no other restraint but the law of
nature.}

\textbf{Civil liberty}. Liberty in society consists in 
\begin{itemize}
    \item rule of law that applies equally to all
    \item laws that are created by representative government
    \item liberty in those things which are not governed by law
    \item freedom from the "arbitrary will of another man"
\end{itemize}

\textbf{Political authority is limited:} the "liberty of man" under government
is subject to the rule of law.

\subsubsection{The creation of political authority}

\textbf{1. The social contract}: people agree in the state of nature to give up
their executive powers to carry out the laws of nature.

\textbf{2. Civil society}: people become a \textit{Community} when they pool
their powers (the right of government comes from civil society).

\textbf{3. Commonwealth}: government is created by the majority of the
Community by placing their powers in it \textit{in trust} (i.e. the government
is an \textit{agent} of the people).

Beginnings of modern liberalism: government is there to serve the people.

The government has its political authority in the form of \textit{trust}(in
both senses\footnote{
    Sense 1: trusting somebody. We need to trust that the government
    actually does its work in the interest of the public good.
    Sense 2: like putting money in the trust fund, then someone manages them on
    our behalf.
}) from the people.

Only the people can judge whether the government serves their interests (whether
they maintain their trust).

A legitimate government respects the law of nature (does not want to enslave
people).

\subsection{Hobbes vs Locke}

\subsubsection{Hobbes}

\textbf{Alienation social contract}: people give up their natural rights; there
are no limits on the political authority of the Sovereign.

People cannot alienate their right to self-defense, which conflicts with the
Sovereign's absolute political authority.

\subsubsection{Locke}

\textbf{Agency social contract}: people retain their natural rights; they put
limits of the political authority of the government.

The government's political authority derives from peoples' consent.

\subsection{Consent and legitimacy}

The legitimacy of political authority rests on consent in two ways:

\begin{itemize}
    \item \textbf{Contractual consent}: agreeing to the social contract to give
    up the executive rights to the law of nature.
        \begin{itemize}
            \item Contractual consent may require \textit{actual consent}
            \item \textsc{Problem}: how to think about this (Who gave consent?)
        \end{itemize}
    \item \textbf{Political consent}: consenting to the right of government to
    exercise the executive power of the law of nature
    \begin{itemize}
        \item Political consent takes the form of \textit{tacit consent}
        (contractual consent implies political consent, because it comes from
        civil society and not individual citizens)
    \end{itemize}
\end{itemize}

\subsection{Property}

\textbf{The origin of property}

\begin{itemize}
    \item Starting point: \textbf{joint ownership} -- God gave the Earth to all
    of mankind in common.
    \begin{itemize}
        \item Problem: explaining how private property can arise without the
        consent of everyone.
    \end{itemize}
    \item Locke's starting point: \textbf{self-ownership} -- everyone has
    property rights in their person (their body, mind, and labor).
    \begin{itemize}
        \item Note the radical implications: if everyone "owns" their own
        person, then the right to life and liberty is a \textit{property
        right}.
        \item Thus, the source of property rights is not the sovereign.
        \item Therefore, \underline{property rights are prior to political
        society}.
    \end{itemize}
\end{itemize}

\textbf{Mixing labor theory}

The original claim to property is based on adding labor to resources. The
justification of property comes from the \textbf{additional value} created by
labor.

\begin{itemize}
    \item It is a \textbf{law of nature} that people are responsible to improve
    upon the world.
    \item The same law of nature forbids wasting or spoiling useful resources.
    \item Since the source of property rights is not the sovereign, rulers do
    not have rights to the property of their subjects (i.e. taxation must be
    consented to).
\end{itemize}

\subsection{The proviso}

\textbf{The Lockean proviso}: the appropriation of property is justified iff
"enough and just as good" is left. That is, a person gets property right only
over resources which she actually uses.

\textbf{Money}: a social convention which makes it possible that inequalities
arising from private property does not violate the laws of nature. It involves
a \textit{tacit agreement} to put value on (and create rights to)
"larger possessions".

%%%%%%%%%%%%%%%%%%%%%%%%%%%%%%%%%%%%%%%%%%%%%%%%%%%%%%%%%%%%%%%%%%%%%%%%%%%%%%%%
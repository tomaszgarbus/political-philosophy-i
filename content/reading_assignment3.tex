\section{Reading assignment 3: Locke - The Second Treatise of Civil
Government}

\subsection{Introductory}

\textit{Political power, then I take to be a right of making laws with
penalties of death, and consequently all less penaltyies, for the regulating
and preservin of property, and for employing the force of the community, in
the execution of such laws (...) all this only for the public good.}

\subsection{Of the State of Nature}

A state of nature is a state of perfect freedom and equality, but not a state
of licence: man has no liberty to destroy himself or any creature except for a
nobler cause.

Everyone is bound to preserve himself and to preserve the rest of mankind.

In the state of nature one man can overcome another by power only as a means
of punishment for the offender's transgressions.

\textit{
In transgressing the law of nature, the offender declares himself to live by
another rule than that of reason and common equity, which is that measure God
has set to the actions of men, for their mutual security; and so he becomes
dangerous to mankind, the tie, which is to secure them from injury and
violence, being slighted and broken by him.
}

Only the person who suffered from the hands of another can remit (forgive), 
the magistrate cannot do it in their name.

The damnified person has the power of appropriating to himself the goods or
service of the offender, by \textit{right of self-preservation}.

\subsection{Of the State of War}

\subsection{Of Slavery}

\subsection{Of Property}

\subsection{Of Political or Civil Society}

\subsection{Of the Beginnings of Political Societies}

\textit{
The same law of nature, that does by this means give us property, does also
bound that property too. God has given us all things richly, is the voice of
reason confirmed by inspiration. But how far has he given it to us? To enjoy.
As much as any one can make use of to any advantage of life before it spoils,
so much he may be his labour fix a property in.
}

Labor gives right of property. For example, enclosing a piece of land to
cultivate it is enough to claim property (if the size of land is not
excessive).

\section{Reading assignment 3: Hampton}

\textbf{Agency social contract theory}: rulers as the people's "employees"
remain under our control.

Lock wrote the \textit{Treatise} for political purpose:
\begin{itemize}
	\item Refute Filmer's divine rights theory
	\item Provide philosophical license for the rebellious activities he
	and his friends had undertaken against the Britisch rulers Charles II
	and James II, which culminated in 1688 in overthrow of the latter in
	what the rebels called the Glorious Revolution
	\item Therefore he is clearly supportive of allowing the "firing" of
	unsatisfactory rulers by dissatisfied subjects.
\end{itemize}

Locke thinks human beings are naturally more other-regarding and more
cooperative than Hobbes takes them to be.

God's "Fundamental Law of Nature" directs people to preserve the life, health
and possessions of others as long as their own preservation will not be
compromised by doing so.

The State of Nature has a Law of Nature to govern it, which obliges every one:
And Reason, which is that Law, teaches all Mankind, who will but consult it,
that being all equal and independent, no one ought to harm another in his Life,
Health, Liberty or Possessions.

Like Hobbes, Locke insists that people are \textit{politically} equal.

Locke argues that in the state of nature, the law to respect others' persons
and possessions would be obeyed by all rational persons.

Warfare is precipitated by irrational members of society who either harm others
for their own gain (\textit{"In trangressing the Law of Nature, the Offender
declares himself to live by another Rule, than that of reason and common
Equity"}) or fail (because of personal bias) to interpret the fundamental law
of nature correctly, especially when they use it to justify the punishment of
offenders.

In an iterated PD Lockean people would behave no differently from Hobbesian
people.

The State should solve three problems (\textit{inconveniences}):

\begin{itemize}
	\item Establish a Law by common consent
	\item Set a known and indifferent Judge with Authority to determine
	all differences according to the Law
	\item Create the Power to back and support the Sentence when right,
	and give it due Execution
\end{itemize}

Unlike Hobbes, Locke's ruler should have clear limits on his authority and
power.

Like Hobbes, Locke makes individual consent the source of all political
authority.

God's laws enable people to have property rights in the state of nature, so
that property is something that is conceptually prior to political society.
The ruler is not the source of property rigts in a society.

Our conception of political societies is deeply tied to the idea that, as Locke
puts it, government has "direct jurisdiction" over land. That idea seems to
conflict with a consent-based justification of political authority as Locke has
formulated it.

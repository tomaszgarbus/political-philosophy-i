\section{Key concepts 7}
\begin{itemize}
	\item \textbf{Pareto optimality, Pareto improvement,Pareto efficiency}:
	 Pareto improvement improves one or more dimensions without degrading
	 any dimension. Pareto optimality is when no further Pareto improvement
	 is possible. Pareto efficiency is the same as Pareto optimality.
	\item \textbf{original position}: state of nature for Rawls
	\item \textbf{veil of ignorance}: a situation where we have to design
	 a just social system without knowing our place in that system (not
	 even our talents, health etc.)
	\item \textbf{moral person}: rational, neither risk-averse nor
	 risk-seeking person with a sense of justice, able to act in accordance
	 to her own conception of the good, not influenced by envy, has a life
	 plan, able to assess her goals from a moral point of view
	\item \textbf{impartiality}: for Rawls it is realized through the
	 veil of ignorance framework
	\item \textbf{primary goods (natural and social)}: natural primary
	 are health, speed, intelligence, imagination etc. Social primary goods
	 are rights, liberties and opportunities, power, income, wealth,
	 (the social bases of) self-respect.
	\item \textbf{circumstances of justice}: people are physically close,
	 they are roughly equal, and their psychological and intellectual
	 abilities are finite. Goods are only moderately available but not
	 extremely scarce. People are only moderately other-regarding.
	\item \textbf{constraints on the concept of right}: only conceptions
	 of justice worth considering must conform to the conditions of:
	 generality, universality, publicity, ordering, finality
	\item \textbf{maximin rule}: choose the conception of justice which
	 maximizes the outcome for the least principled member
	\item \textbf{total and average utilitarianism}: same as classical and
	 modern
	\item \textbf{Principle of Liberty}: each person has an equal right to
	 the most extensive schemes of basic liberties that is compatible with
	 a similar scheme of liberties for others
	\item \textbf{Principle of Fair Equality of Opportunity}: social and
	 economic inequalities are to be arranged so that they are attached to
	 positions and offices open to all
	\item \textbf{fair versus formal equality of opportunity}: fair
	 equality of opportunity is more demanding then formal because it also
	 takes into account (besides talent) socioeconomical standings etc.
	\item \textbf{Difference Principle}: social and economic inequalities
	 are to be arranged so that they are reasonably expected to be to
	 everyone's advantage
	\item \textbf{lexicographic ordering}: first of the two principles
	 (that is, the Principle of Liberty) takes priority
\end{itemize}

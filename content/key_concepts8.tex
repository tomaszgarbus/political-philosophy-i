\section{Key concepts 8}

\begin{itemize}
	\item \textbf{the individualism objection}: liberalism leads to
	 individualism, atomism and the destruction of traditional social
	 structures
	\item \textbf{the idea of the end of history}: liberal democracy may be
	 the end point of mankind's indeological evolution and as such
	 considered the end of history
	\item \textbf{Lockean vs Rawlsian liberal tradition}: Locke focuses on
	 the danger to liberty from government, Rawls also on equality. Locke
	 advocates a minimal small government, Rawls typically advocates an
	 active welfare state. Locke emphasizes the rights and liberties of
	 citizens, Rawls' two principles are an attempt to unify liberty and
	 equality by prioritizing liberty but heavily considering
	 distributive organization of society too.
	\item \textbf{perfectionist liberalism}: (Mill) the fundamental values
	 of liberal theory are matters of moral truth, liberalism is justified
	 because it is morally the best
	\item \textbf{political liberalism}: (Rawls) the theory has no
	 commitments regarding the moral truth of its fundamental values.
	 Liberalism is justified because it embodies the shared values ("public
	 reason") of citizens
	\item \textbf{the principle of state neutrality}: political
	 institutions must have neutral justifications (that is, independent
	 from any conception of the good)
	\item \textbf{main features of communitarian theories}: people cannot
	 be conceived as presocial because their identities are determined by
	 the community to which they belong. Societies are based on established
	 and fixed identities. Liberals build upon an implausible and
	 incoherent concept of the person. Human beings are inherently social
	 and political theorizing must take this into account.
	\item \textbf{liberal responses to the communitarian challenge}:
	 liberalism does not deny the social structure and embeddedness of
	 human beings. It does insist that moral justification is owed to
	 every member of society. Social circumstances do not fully determine
	 the identity of human beings. Communitarians cannot critical distance
	 from the institutions and practices by which they evaluate them.
	 Communitarian ideas might be unjust or morally objectionable.
	\item \textbf{fallibilism}: we can never have certainty about empirical
	 facts, we can only falsify but not prove theories
	\item \textbf{open and closed societies}: closed societies are
	 authoritarian and use tradition or future utopian vision as
	 justification. Civil liberties and civil society are limited,
	 knowledge is political. Little or no freedom of thought and
	 expression. Open societies are the opposite, they allow to have a
	 critical attitude towards authority and tradition. The distinction is
	 epistemological rather than political. Open societies accept that all
	 knowledge is provisional.
\end{itemize}

\section{Lecture 8: Contemporary liberalism and its critics}

\subsection{The end of history?}

Francis Fukuyama, The End of History and the Last Man (1992):
\textit{Liberal democracy may constitute the "end point of mankind's
ideological evolution" and the "final form of human government" and as such
constituted the "end of history". That is, while earlier forms of government
were characterized by grave defects and irrationalities that led to their
eventual collapse, liberal democracy was arguably free from such fundamental
internal contradictions}.

\subsection{The individualism objection, again}

\textbf{The individualism objection}

Liberalism leads to individualism, atomism and the destruction of traditional
social structures.

\textbf{Liberalism}

The ultimate aim of politics is to secure the conditions for the flourishing of
\textbf{individuals}.

\textbf{Communitarianism}

The ultimate aim of politics is the protection or realization of some kind of
ideal \textbf{community}.
\begin{itemize}
	\item There are many forms of communitarianism; it is more a label for
	 disparate non-liberal views than any kind of consistent political
	 philosophy.
	\item Some forms of communitarianism are democratic and respect
	 individual rights.
	\item Others are more authoritarian even if they may (only) claim to
	 respect democracy.
\end{itemize}

\subsection{Types of liberalism}

\textbf{Lockean tradition}
\begin{itemize}
	\item Focus is on the danger to liberty from government.
	\item Typically advocates a minimal (small) government.
	\item Emphasizes the rights and liberties of citizens.
\end{itemize}

\textbf{Rawlsian tradition}
\begin{itemize}
	\item Equality is also important.
	\item Rawls' two principles may be seen as an attempt to unify liberty
	 and equality by giving primacy to liberty but giving a role to
	 distributive considerations as well.
	\item Typically advocates an active welfare state.
\end{itemize}

\subsection{Taking stock}

\textbf{The central problems of politics}

\begin{enumerate}
	\item \textbf{Political authority}
	\begin{itemize}
		\item How can political authority and political obligations be
		 justified?
		\item What makes the government morally entitled to rule (i.e.
		 legitimate)?
	\end{itemize}
	\item \textbf{Justice}
	\begin{itemize}
		\item What is the extent of political authority? What are the
		 limits of state interference?
		\item How can the benefits of social cooperation be fairly
		 distributed?
	\end{itemize}
	\item \textbf{Toleration}
	\begin{itemize}
		\item How can it be ensured that people with different
		 conceptions of the good (i.e., basic moral and metaphysical
		 views) can get along with one another?
		\item How can the government's actions and policies be
		 justified such that they are acceptable to all? Can the state
		 enforce morality?
	\end{itemize}
\end{enumerate}

\subsection{Toleration}

The state should not prescribe or prefer any sort of moral or metaphysical
(religious) view of life's meaning and value (i.e. conception of the good) --
that is, it should not prescribe how citizens should live their lives or form
their own conception of the good.

\textbf{Two kinds of liberalism}:
\begin{itemize}
	\item \textbf{Perfectionist liberalism}: the fundamental values of
	 liberal theory (autonomy, liberty, well-being, ...) are matters of
	 \textbf{moral truth}; liberalism is justified because it is the
	 morally best system of government (Mill).
	\item \textbf{Political liberalism}: the theory has no commitments
	 regarding the moral truth of its fundamental values; liberalism is
	 justified because it embodies the shared values ("public reason")
	 of citizens (Rawls).
\end{itemize}

\subsection{Political liberalism}

Rawls:
\textit{The conception of justice should be, as far as possible, independent
of the opposing and conflicting philosophical and religious doctrines that
citizens affirm. In formulating such a conception, political liberalism
applies the principle of toleration to philosophy itself. The religious
doctrines that in previous centuries were the professed basis of society have
gradually given way to principles of constitutional government that all
citizens, whatever their religious view, can endorse.}

\begin{itemize}
	\item The justification of liberal constitutional democracy must be
	 independent of any conception of the good.
	\item Toleration requires that the state remains neutral between
	 different conceptions.
	\item \textbf{Neutrality} is a basic principle on this view.
	\item In practice: political institutions and policies must have
	 \textbf{neutral justifications}.
\end{itemize}

\subsection{Communitarianism}

\textbf{The individualism objection}

Liberalism leads to individualism, atomism and the destruction of traditional
social structures.

\textbf{The central communitarian thesis}

People cannot be conceived as \textbf{presocial} because their identities are
determined by the community to which they belong. Societies are based on
established traditions and fixed identities (e.g., the family as a model of a
greater good than its members' good).
\begin{itemize}
	\item Human beings are inherently social, "embedded" in social
	 practices with specific identities, roles and obligations.
	\item Political theorizing must be carried out in the context of the
	 community and its traditions.
	\item Liberals build upon an implausible and incoherent concept of the
	 person (e.g. Rawls).
\end{itemize}

\textbf{Liberal responses}
\begin{itemize}
	\item Liberalism does not deny the social structure and embeddedness of
	 human beings.
	\item It does insist that moral justification is owed to each
	 individual and therefore the individual must be the starting point of
	 political theorizing.
	\item The identity of human beings is not fully determined by their
	 social circumstances and traditions.
	\item Communitarians want to evaluate institutions and practices on the
	 basis of ideas generated by those very institutions and practices,
	 thus they cannot maintain a critical distance from them.
	\item Those ideas might be unjust or morally objectionable.
\end{itemize}

\subsection{Science and democracy}
Carl Sagan (1996)
\textit{
We’ve arranged a society based on science and technology in which nobody
understands anything about science and technology, and this combustible
mixture of
ignorance and power, sooner or later, is going to blow up in our faces.
I mean, who is
running the science and technology in a democracy if the people don’t know
anything
about it? And the second reason that I’m worried about this is that science is
more than a
body of knowledge. It’s a way of thinking. A way of skeptically interrogating
the universe
with a fine understanding of human fallibility. If we are not able to ask
skeptical
questions, to interrogate those who tell us that something is true, to be
skeptical of those
in authority, then we’re up for grabs for the next charlatan, political or
religious, who
comes ambling along. It’s a thing that Jefferson laid great stress on. It
wasn’t enough, he
said, to enshrine some rights in a Constitution or a Bill of Rights. The
people had to be
educated, and they had to practice their skepticism and their education.
Otherwise we
don’t run the government—the government runs us.}

\subsection{Karl R. Popper (1902-1994)}
\begin{itemize}
	\item Born in Vienna in an upper middle class Jewish family (that
	 converted to Lutheranism).
	\item 1918-1919 attends university as a guest student, brief
	 association with Marxist student movement
	\item 1922-1928 formal university student, Doctorate in Psychology
	\item 1934 \textbf{The logic of scientific discovery}
	\item 1937 emigrates to New Zealand, teaches in Christchurch
	\item 1945 \textbf{The open society and its enemies}
	\item 1946- moves to London School fo Economics
	\item 1957 \textbf{The povery of historicism}
	\item 1965 knighted
\end{itemize}

\subsection{The demarcation problem: how to distinguish science from
pseudo-science?}
\textbf{Falsificationism}: a hypothesis is scientific iff it can potentially be
refuted by some possible observation.

\begin{itemize}
	\item Falsifiability was originally proposed as a criterion of
	 hypothesis testing in science.
	\item The criterion for distinguishing scientific and pseudo-scientific
	 theorems turns out to be the very same criterion that can be used for
	 evaluating scientific hypothesis!
\end{itemize}

\textbf{Fallibilism}: we can never have certainty about empirical facts. All
that any observational test (observation or experiment) can do is to show that
a theory is false.

\subsection{Scientific knowledge}

\textbf{Science}: the rigorous, self-correcting application of general
epistemic principles to discover epistemic reasons.
\begin{itemize}
	\item We can never have certainty about empirical facts: all that
	 any observation or experiment can do is to show that a hypothesis is
	 false.
	\item Scientific knowledge is always provisional.
	\item Scientific knowledge presupposes a critical attitude -- including
	 a critical attitude towards authority and tradition.
	\item Scientific knowledge presupposes freedom of thought and
	 expression.
\end{itemize}

\subsection{Open and closed societies}
\textbf{Closed societies}
\begin{itemize}
	\item Traditional societies are closed: there is no critical attitude
	 towards tradition, because there is no distinction between natural
	 laws and convetions (the prevailing order is "natural").
	\item Questioning authority and tradition is suppressed and
	 controlled.
	\item There is little or no freedom of thought and expression.
	\item Civil liberties and civil society are limited.
	\item Closed societies make knowledge political.
\end{itemize}

\textbf{Open societies}
\begin{itemize}
	\item Open societies begin to appear when the distinction between
	 natural laws (to be discovered by science) and conventional laws
	 (to be argued for and justified) is recognized.
	\item The distinction makes it possible to have a \textbf{critical
	 attitude} towards authority and tradition.
	\item Open societies are the political manifestation of accepting that 
	 all knowledge is provisional.
	\item Open societies necessarily accept value pluralism and freedom of
	 thought and expression.
	\item The distinction between open and closed societies is
	 \textbf{epistemological} rather than political.
\end{itemize}

\subsection{All authoritatian societies are closed societies}
\textbf{Authoritarianism}
\begin{itemize}
	\item Authoritarianism aims to restore (an imaginary) glorious past
	 period or realize an ideal in the future (utopianism).
	\item Conventional laws are claimed to be "natural".
	\item There is little tolerance tof a critical attitude towards
	 tradition.
	\item Authoritarianism rejects value pluralism and aims to suppress
	 freedom of thought and expression.
	\item Authoritarians attack science and expertise -- they \textbf{make
	 knowledge political}.
\end{itemize}

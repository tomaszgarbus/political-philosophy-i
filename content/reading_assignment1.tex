\section{Reading assignment 1: Plato - Crito}

Socrates' friend, Crito, soon before his planned execution, to attempt one last
time to convince Socrates to accept help from his friends in escaping the
prison.

Crito cites three reasons why Socrates should accept. Firstly, he is a beloved
friend who cannot be replaced. Moreover, if Socrates is executed, Crito will
be disgraced in society's eyes because people will think that Crito either
couldn't or didn't want to save him. Finally, Socrates' children will be left
as orphans.

There begins Socrates' explanation why he refuses to escape his penalty.

Firstly, he preaches to Crito that he should not be concerned with the public
opinion. Good men will know things as they are, and the others' opinion is
irrelevant. He uses multiple analogies, for example that of a gymnast and their
trainer. The athlete should pay attention to their coach's criticism but not
to anyone else's.

Then he presents arguments against escaping:

By escaping, he would undermine the authority law and the state. He would be 
setting an example for other Athenians to ignore the law and the collective
decisions of direct democracy.

He claims that Athenians have a moral obligation to be obedient to the state
out of gratitude for what they have already received: an upbringing, education.

Socrates also brings up the fact that every free Athenian, once they come of
age, is free to take their belongings and leave where they wish. Therefore,
by staying in Athens, they have implicitly accepted the social contract
embodied by the law.

For Socrates, there are two acceptable stances in relation to the law:
either obey the commands, or convince your co-citizens that the commands are
wrong. Disobedience is out of question and in his eyes it is wrong in three
ways: it wrongs ones' parents, it wrongs ones' teachers and it breaks the
social contract.

Finally, Socrates states that by escaping, he would confirm the verict of
the judges, who deemed him as someone who corrupts the youth, someone who
destroys the laws. By accepting his sentence, he will prove them wrong and show
his virtue and justice.

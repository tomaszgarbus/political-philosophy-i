%%%%%%%%%%%%%%%%%%%%%%%%%%%%%%%%%%%%%%%%%%%%%%%%%%%%%%%%%%%%%%%%%%%%%%%%%%%%%%%

\section{Lecture 5: Classical liberalism -- Mill}

There is a lot of misunderstanding on what liberalism actually is.

\subsection{The main argument}

\begin{enumerate}
    \item The best institutions are those under which people are happiest.
    \item A precondition of happiness is self-development.
    \item Self-development requires free individual experimentation in living.
    \item The states that best permit and promote free individual
    experimentation are liberal states.
    \item Therefore, people will be happiest under liberal states.
    \item Therefore, liberal states provide best institutions.
\end{enumerate}

This is an empirical argument (pointing to actual liberal states).

\subsection{Introduction}

Rousseau has influence on the French revolution, particularly on Jacobins who
took power.

The Reign of Terror (September 1793 -- July 1974)

Robespierre, 5 February 1794: \textit{If virtue be the spring of a popular
government in times of peace, the spring of that government during a revolution
is \textbf{virtue combined with terror} [...] The government in a revolution
is the \textbf{despotism of liberty} against tyranny.}

\textit{Society owes protection only to peaceful citizens; the only citizens
in the Republic are the republicans. For it, the royalists, the conspirators
are only strangers or, rather, enemies. This terrible war waged by liberty 
against tyranny -- is it not indivisible? Are the enemies within not the
allies of the enemies without?}

The French revolution starts by the promise of popular sovereignty and loss of
belief in divine source of monarchical power, but it turns into reign of
terror.

\subsection{The problem of popular sovereignty}

\begin{itemize}
    \item Before the French Revolution: main problem is justifying political
    authority.
    \item After the revolution: justifying the \textbf{extent} of political
    authority.
    \item \textbf{Liberal answer}: inviolable individual rights,
    constitutionalism.
    \item \textbf{Conservative answer}: traditional hierarchy as bulwark
    against unlimited popular sovereignty.
    \item \textbf{The individualism objection}: liberalism leads to
    individualism, atomism and the destruction of traditional social
    structures\footnote{This is the conservatives' objection against
    liberalism}.
\end{itemize}

Constitutions are by design difficult to change and that is their point.

Conservative authors are typically concerned with particular countries or
times, but there are no major general conservative theories.

What passes today for conservative politics is very far from the traditional
conservative answers.

\subsection{The liberal answer}

\textbf{Declaration of the Rights of Man and of the Citizen (1789)}

\textit{Political Liberty consists in the power of doing whatever does not
injure another. Thus the exercise of the natural rights of every man has no
other limits than those which are necessary to secure to every other member
of society enjoyment of the same rights.}

\textbf{Mill's Liberty principle (or Harm principle) (1859)}

\textit{The sole end for which mankind are warranted, individually or
collectively in interfering with the liberty of action of any of their number,
is self-protection. That the only purpose for which power can be rightfully
exercised over any member of a civilized community, against his will, is to
prevent harm to others. His own good, either physical or moral, is not
a sufficient warrant.}

\subsection{John Stuart Mill (1806-1873)}

\begin{itemize}
    \item Son of James Mill, educated at home with the aim of creating
    a genius
    \item Starts Greek at 3, Latin at 7, reads Aristotle in the original by 10
    \item 1826 nervous breakdown and recovery
    \item 1851 marries Harriet Taylor
    \item 1859 \textbf{On Liberty}
    \item 1861 \textbf{Considerations on Representative Government}
    \item 1863 \textbf{Utilitarianism}
    \item 1865-1868 Member of Parliament
    \item 1866 calls for granting women the right to vote
\end{itemize}

\subsection{The Liberty principle}

\textit{The sole end for which mankind are warranted, individually or
collectively in interfering with the liberty of action of any of their number,
is self-protection. That the only purpose for which power can be rightfully
exercised over any member of a civilized community, against his will, is to
prevent harm to others. His own good, either physical or moral, is not
a sufficient warrant.}

Implications of the Liberty Principle

\begin{itemize}
    \item \textbf{Anti-paternalism} Interference with a person's liberty or
    freedom of action for the sake of promoting her own good is impermissible,
    even if the person is acting against her own good.
    \item \textbf{Harm to others} Interference with joint activities of
    consenting adults is impermissible as long as they do not cause any harm
    to third parties, even if they cause harm to one another.
    \item The principle entails that the \textbf{burden of proof} is always on
    those who want to restrict liberty.
\end{itemize}

\textbf{Example: motorcycle helmet and seatbelt laws}

The Liberty principle may seem to exclude some forms of \textbf{paternalism}
that seem justified.

\begin{itemize}
    \item Mill might respond that not wearing a helmet is a sign of
    irrationality. We should distinguish between:
    \begin{itemize}
        \item \textbf{Soft paternalism}: interference with a person's liberty
        for the sake of that person's good when the person is irrational,
        uninformed or incompetent in some way
        \item \textbf{Hard paternalism}: interference with a person's liberty
        for the sake of that person's good when the person is fully informed
        and competent
    \end{itemize}
    \item Another response is that third parties are harmed when the medical
    expenses of reckless drivers must be paid by them, since even reckless
    drivers are going to be treated for humanitarian reasons.
\end{itemize}

\textbf{Example: public nuisances}

Public behavior may sometimes be prohibited not only on the basis of its
harmfulness, but also its \textbf{intrusiveness} ("a violation of good
manners")
\begin{itemize}
    \item e.g the use of mobile phones during lecture does not literally
    harm anyone, but it's intrusive and annoying: it intrudes on a public space
    \item how to distinguish between \textbf{permissible} and
    \textbf{impermissible} intrusions?
    \item e.g. can an annoying mass protest in a public space be prohibited?
    \item the distinction is made in terms of the \textbf{context} of the
    activity
\end{itemize}

\subsection{Utilitarianism}

\textit{I forego any advantage which could be derived to my argument from the
idea of abstract right as a thing independent of utility. I regard utility
as the ultimate appeal on all ethical questions; but it must be utility in the
largest sense, grounded on the permanent interests of man as a progressive
being.}

\textbf{Consequentialism}

Actions (policies, etc.) are evaluated solely by the value (goodness) of
their consequences.

\textbf{Utilitarianism}

The right action (policy, etc.) is that which maximizes the well-being of the
affected individuals.

\begin{itemize}
    \item Bentham: "the greatest happiness of the greatest number"
    \item Mill: well-being consists in happiness and happiness consists in
    pleasure
    \item a mathematical representation of well-being
    \item sum-ranking: the value of every outcome is determined exclusively
    by aggregate utilities
\end{itemize}

\begin{tabular}{c|c c c c c c}
     & $x_1$ & $x_2$ & $x_3$ & $x_4$ & ... & $x_m$ \\
     \hline
     1 & $u_1(x_1)$ & $u_1(x_2)$ & $u_1(x_3)$ & $u_1(x_4)$ & ... & $u_1(x_m)$\\
     2 & $u_2(x_1)$ & $u_2(x_2)$ & $u_2(x_3)$ & $u_2(x_4)$ & ... & $u_2(x_m)$\\
     \vdots & \vdots & \vdots & \vdots & \vdots & ... & \vdots \\
     n & $u_n(x_1)$ & $u_n(x_2)$ & $u_n(x_3)$ & $u_n(x_4)$ & ... & $u_n(x_m)$\\
\end{tabular}

$U(x_j) = u_1(x_1) + u_2(x_j) + u_3(x_j) + ... + x_n(x_j) =
\sum_{i=1}^{n} u_i(x_j)$

\subsection{Mill on well-being}

There is pleasure and pain, happiness consists of getting pleasure and
avoiding pain.

\textbf{Higher and lower pleasures}

\textit{It is quite compatible with the principle of utility to recognize the
fact, that some kinds of pleasure are more desirable and more valuable than
others. It would be absurd that while, in estimating all other things, quality
is considered as well as quantity, the estimation of pleasures should be
supposed to depend on quantity alone.}

\textbf{The competent judges test}

\textit{If I am asked what I mean by difference of quality in pleasures, or
what makes one pleasure more valuable than another, merely as a pleasure,
except its being greater in amount, there is but one possible answer. If one
of the two is, by those who are competently acquainted with both, placed so
far above the other that they prefer it [...] we are justified in ascribing
to the preferred enjoyment a superiority in quality so far outweighing
quantity as to render it, in comparison, of small account.}

\textbf{Socrates and the fool}

\textit{It is better to be a human being dissatisfied than a pig satisfied;
better to be Socrates dissatisfied than a fool satisfied. And if the fool, or
the pig, are of a different opinion, it is because they only know their own
side of the question.}

\textbf{Pursuing happiness indirectly}

\textit{Those only are happy (I thought) who have their minds fixed on some
object other than their own happiness; on the happiness of others, on the
improvement of mankind, even on some art or pursuit, followed not as a means,
but as itself an ideal end. Aiming thus at something else, they find
happiness by the way.}

\subsection{Liberalism and well-being}

\begin{itemize}
    \item At the center of Mill's conception of well-being is the capacity
    for \textbf{self-development} according to one's own direction and plan
    of life.
    \item The conception of the human good is rooted in the idea of
    \textbf{progress} (both individual and social):
    \begin{itemize}
        \item Social institutions should enable self-development and thereby
        human flourishing.
        \item Social institutions should allow as many forms of
        self-development as possible.
        \item A prerequisite of social progress are \textbf{experiments in
        living}: the free development of individuality.
        \item Humans are \textbf{progressive beings}.
        \item The ultimate ideals are \textbf{autonomy} and
        \textbf{self-determination}.
        \item Rights are based on such central, permanent human interests.
    \end{itemize}
\end{itemize}

\subsection{The answer to the individualism objection}

\textit{Human beings owe to each other help to distinguish the better from
the worse, and encouragement to choose the former and avoid the latter. They
should be forever stimulating each other to increased exercise of their
higher faculties.}

\begin{itemize}
    \item There is an obligation to help the self-development of others
    through exhortation and persuasion, but not compulsion and coercion.
    \item Therefore, you have no moral obligation to develop your own virtues,
    but you do have a moral obligation to help others develop them!
    \item It is \textbf{never permissible} to interfere with the liberty of
    others in the development of their virtues and pursuit of their good, no
    matter how mistaken they might be about them.
    \item It is only self-development and autonomy that make it possible to
    discover and enjoy the higher pleasures.
\end{itemize}

%%%%%%%%%%%%%%%%%%%%%%%%%%%%%%%%%%%%%%%%%%%%%%%%%%%%%%%%%%%%%%%%%%%%%%%%%%%%%%%